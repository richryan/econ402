% =====================================================================
% Program: econ-402-ec2.tex
% Purpose: Suggested solutions to the second extra credit.
%
% E-mail: RichRyan@umich.edu
% =====================================================================
\documentclass[12pt]{pracjourn_rwr}
% \documentclass[12pt]{article}

\usepackage[dvipsnames]{xcolor}
\usepackage{url}
\urlstyle{same}
\usepackage{booktabs}
\usepackage{graphicx}
\usepackage{amsmath,amsthm,amssymb}
\usepackage{amsxtra}
\usepackage{natbib}
\usepackage[left=1in,right=1in,top=1in,bottom=1in]{geometry}
\usepackage{alltt}
\input{ee.sty}
\usepackage{listings}
% \usepackage[inline]{showlabels}
\usepackage{paralist} % inline lists (enumeration)
\usepackage{mathabx} % widebar

% Set new counter to start everything off with a "2" for section 2
\newcounter{rich}
\setcounter{rich}{2}
\newcounter{ryan}
\setcounter{ryan}{5}
\renewcommand{\theryan}{\Alph{ryan}\arabic{rich}}

% Change figure numbering
\usepackage{chngcntr}
\counterwithin{figure}{ryan}

% \newtheorem{theorem}{Theorem}[chapter]
% \newtheorem{lemma}{Lemma}[chapter]
\newtheorem{theorem}{Theorem}[rich]
\newtheorem{lemma}{Lemma}[rich]

% \newtheoremstyle{definition}% Name
% {3pt}% Space above
% {3pt}% Space below
% {}% Body font
% {}% Indent amount
% {\bfseries}% Theorem head font
% {:}% Punctuation after theorem head
% {.5em}% Space after theorem head
% {}% Theorem head spec (can be left empty, meaning "normal")

% \setcounter{chapter}{2}
\theoremstyle{definition}
\newtheorem{definition}{Definition}[rich]
\newtheorem{example}{Example}[rich]
\newtheorem{exercise}{Exercise}[rich]
\newtheorem{prop}{Proposition}[rich]
\newtheorem{assumption}{Assumption}[rich]
\newtheorem{fact}{Fact}[rich]

\theoremstyle{remark}
\newtheorem{remark}{Remark}[ryan]
% \newtheorem{remark}{Remark}[chapter]
% \newtheorem{fact}{Fact}[chapter]

% === Change to homework numbering ===
\renewcommand{\thesection}{E\arabic{rich}.\arabic{section}}

\title{Suggested solution for the second Extra Credit}
% \author{Rich Ryan}
\email{RichRyan@umich.edu}
%% \TPJrevision{2013}{08}{03} % The editors ask you to update this for each %% revision submitted
% \TPrevision{3}{1}{2015}
\license{\texttt{econ-402-ec2.tex}} % May be used if you want a copyright notice at the bottom right of the first page
\abstract{Time-varying prices of capital.}
\newinfo\datestart{Date Started}
\newinfo\daterevised{Date Revised} % See the pracjourn class for how to do this
\datestart{March 7, 2018}
\daterevised{March 17, 2018}


% === Spacing in the table of contents
\usepackage{tocloft}
\renewcommand{\cftsecnumwidth}{3em}
\renewcommand{\cftsubsecnumwidth}{5em}
\renewcommand{\cftsecleader}{\cftdotfill{\cftdotsep}}
% Set TOC depth
\setcounter{tocdepth}{2}


\begin{document}
\maketitle

\tableofcontents

\section{Economic environment}

Time is discrete and indexed by $t = 0,1,2,\dots$
Two types of firms populate the economic environment.\footnote{At least this is one motivation for the user-cost-of-capital formula.}
The first type produces the final consumption good using labor and capital as inputs.
The second type optimally accumulates capital that is rented out to final-good firms.

Consider a capital-leasing firm's decision to purchase a unit of capital.
How much would the firm be willing to pay for the unit of capital?
When a capital-leasing firm purchases a unit of capital, the capital comes online the following period.
The firm is able to rent this out to a final-good firm \emph{and} sell the undepreciated portion of capital.
Because those payments come the following period they need to be discounted to the present using the interest rate $r$.
In other words,
\begin{equation}
\label{eq:1}
p_{t} = \frac{R_{t+1} + (1-\delta)p_{t+1}}{1+r},
\end{equation}
where $p_{t}$ is the price of new capital, $R_{t+1}$ is the rental rates, $\delta$ is the depreciation rates, and
$r$ is the interest rate.
See example 4.7 in the \href{https://umich.app.box.com/s/dlnc24jda0xu9sqifl2uicc4cdmzit4l}{notes}.

Final-good-producing firms have access to a neoclassical production technology defined in definition 4.1.
Final-good-producing firms face a tax on their profits $\tau_{t}$, which varies through time.
At any point $t$ the final-good-producing firms maximize profits by solving
\begin{align*}
\max_{K,N} \; (1-\tau_{t}) \left[ F(K,N) - w_{t}N \right] - R_{t}K .
\end{align*}
The firm's first-order conditions with respect $K$ and $N$ are
\begin{align*}
0 &= (1-\tau_{t})F_{K} - R_{t} \\
0 &= (1-\tau_{t})\left[ F_{N} - w_{t} \right].
\end{align*}
Rearranging the first yields
\begin{equation}
\label{eq:2}
R_{t} = (1-\tau_{t}) \text{MPK}_{t},
\end{equation}
where $\text{MPK}_{t} := F_{K}$ is the marginal product of capital.
Equation \eqref{eq:2} holds for all $t$.
See example 4.6.\footnote{In reality, the corporate tax code allows firms to write off debt and depreciation among other items,
  not just the wage bill.}

\begin{remark}
\label{remark:1}
Just fyi, the first-order condition for labor is
\begin{equation}
\label{eq:7}
F_{n} = w_{t}.
\end{equation}
If capital is fixed in the short run,
then equation \eqref{eq:7} says that a firm's choice of labor in unaffected by the tax change.
This idea is expressed on page 3 of the \texttt{Lecture 16\_handouts} in the slide titled
``Capital Taxation---Short Run.''
\end{remark}

Combining \eqref{eq:1} with \eqref{eq:2} leads to the optimality condition for capital
\begin{equation}
\label{eq:3}
p_{t} = \frac{(1-\tau_{t+1}) \text{MPK}_{t+1} + (1-\delta)p_{t+1}}{1+r}.
\end{equation}
Equation \eqref{eq:3} holds for all $t$.

\section{Part a) The price of new capital equals the stream of marginal products produced}

Installing a unit of capital entitles the owner to the stream of marginal products produced by the installed unit of capital.
To see this use equation \eqref{eq:3} to recursively substitute in for $p_{t+1}$:
\begin{align*}
p_{t} &= \frac{(1-\tau_{t+1}) \text{MPK}_{t+1} + (1-\delta)p_{t+1}}{1+r} \\
&= \frac{(1-\tau_{t+1}) \text{MPK}_{t+1}}{1+r} + \frac{1-\delta}{1+r}
\left[ \frac{(1-\tau_{t+2}) \text{MPK}_{t+2} + (1-\delta)p_{t+2}}{1+r} \right] \\
&= \frac{(1-\tau_{t+1}) \text{MPK}_{t+1}}{1+r} + \frac{1-\delta}{1+r} \frac{(1-\tau_{t+2}) \text{MPK}_{t+2}}{1+r} +
\left( \frac{1-\delta}{1+r}  \right)^{2} p_{t+2}.
\end{align*}
where the second equality iterates equation \eqref{eq:3} one period forward.
Once more for good measure:
\begin{align*}
p_{t} &= \frac{(1-\tau_{t+1}) \text{MPK}_{t+1}}{1+r} + \frac{1-\delta}{1+r} \frac{(1-\tau_{t+2}) \text{MPK}_{t+2}}{1+r} \\
&\quad + \left( \frac{1-\delta}{1+r}  \right)^{2} p_{t+2} \\
&= \frac{(1-\tau_{t+1}) \text{MPK}_{t+1}}{1+r} + \frac{1-\delta}{1+r} \frac{(1-\tau_{t+2}) \text{MPK}_{t+2}}{1+r} \\
&\quad + \left( \frac{1-\delta}{1+r}  \right)^{2} \left[ \frac{(1-\tau_{t+3}) \text{MPK}_{t+3} + (1-\delta)p_{t+3}}{1+r} \right] \\
&= \frac{(1-\tau_{t+1}) \text{MPK}_{t+1}}{1+r} + \frac{1-\delta}{1+r} \frac{(1-\tau_{t+2}) \text{MPK}_{t+2}}{1+r} \\
&\quad + \left( \frac{1-\delta}{1+r}  \right)^{2} \frac{(1-\tau_{t+3}) \text{MPK}_{t+3}}{1+r}
+ \left( \frac{1-\delta}{1+r}  \right)^{2} \frac{(1-\delta)p_{t+3}}{1+r}.
\end{align*}

This is a pattern.
In general
\begin{align}
\label{eq:4}
\begin{split}
p_{t} &= \frac{(1-\tau_{t+1}) \text{MPK}_{t+1}}{1+r} + \frac{1-\delta}{1+r} \frac{(1-\tau_{t+2}) \text{MPK}_{t+2}}{1+r} \\
&\quad + \left( \frac{1-\delta}{1+r}  \right)^{2} \frac{(1-\tau_{t+3}) \text{MPK}_{t+3}}{1+r}
+ \left( \frac{1-\delta}{1+r}  \right)^{3} \frac{(1-\tau_{t+4}) \text{MPK}_{t+4}}{1+r} + \cdots
\end{split}
\end{align}
Equation \eqref{eq:4} says that the value of installing the capital equals the discounted sum of after-tax incomes generated by installing the unit of capital.
The income earned equals what the firm can rent the capital out for.
The future is discounted by both the interest rate and the decay rate $1-\delta$ (there's less capital to rent out in the future).

\section{Part b) When the marginal product of capital is constant}

When the marginal product of capital is constant equation \eqref{eq:4} becomes:\footnote{Kind Professor House is letting us check our result---how generous.}
\begin{align*}
p_{t} &= \frac{(1-\tau_{t+1}) m}{1+r} + \frac{1-\delta}{1+r} \frac{(1-\tau_{t+2}) m}{1+r} \\
&\quad + \left( \frac{1-\delta}{1+r}  \right)^{2} \frac{(1-\tau_{t+3}) m}{1+r}
+ \left( \frac{1-\delta}{1+r}  \right)^{3} \frac{(1-\tau_{t+4}) m}{1+r} + \cdots
\end{align*}
Simplifying yields
\begin{equation}
\label{eq:5}
p_{t} = \frac{1}{1+r} \left\{
(1-\tau_{t+1})m + \left( \frac{1-\delta}{1+r} \right) (1-\tau_{t+2})m + \left( \frac{1-\delta}{1+r} \right)^{2}(1-\delta) m + \cdots
\right\},
\end{equation}
which is what we were asked to show.

\subsection{Part c) The price of short-lived vs long-lived capital}

Suppose $\tau$ is constant.
The relationship in \eqref{eq:5} becomes
\begin{align*}
p_{t} &= \frac{1}{1+r} \left\{
(1-\tau)m + \left( \frac{1-\delta}{1+r} \right) (1-\tau)m + \left( \frac{1-\delta}{1+r} \right)^{2}(1-\delta) m + \cdots
\right\} \\
&= \frac{(1-\tau)m}{1+r} \left\{
1 + \left( \frac{1-\delta}{1+r} \right) + \left( \frac{1-\delta}{1+r} \right)^{2} + \cdots
\right\} \\
&= \frac{(1-\tau)m}{1+r} \frac{1}{1 - \gamma},
\end{align*}
where the last line uses proposition 4.1 with $\gamma := (1-\delta)/(1+r)$
[and assuming $-1 < (1-\delta) / (1+r) < 1$].
Simplifying this relationship by multiplying the top and bottom by $1+r$ yields
\begin{align}
\label{eq:6}
\begin{split}
p_{t} &= \frac{(1-\tau)m}{1+r} \frac{1+r}{1+r - (1-\delta)} =
\frac{(1-\tau)m}{1+r} \frac{1+r}{r + \delta} \\
&= \frac{(1-\tau)m}{r+\delta}.
\end{split}
\end{align}

Now suppose $\tau = .35$, $r = .05$, and $m = 1$.
We're interested in computing $p_{t}$ when $\delta = .05$ (type I) and $\delta = .25$ (type II).
Plug these numbers in \eqref{eq:6} [in both cases $-1 < (1-\delta) / (1+r) < 1$]:
\begin{align*}
p^{I}_{t} &= \frac{1-.35}{.05 + .05} = \frac{.65}{.1} = 6.5 \\
p^{II}_{t} &= \frac{1-.35}{.05 + .25} = \frac{.65}{.3} = \frac{.65}{.1} \frac{1}{3} = \frac{6.5}{3} = 2 + 1/6.
\end{align*}

The only difference between $p_{t}^{I}$ and $p_{t}^{II}$ is the depreciation rate associated with the capital.
Both types of capital generate the same income stream ($m = 1$).
It is not surprising then that $p_{t}^{I} > p_{t}^{II}$ since type-II capital lasts longer.

\section{Part d) Phased-in tax cut}

Suppose that at time $t$ the tax rate is $\tau_{t} = .3$.
And it is announced that taxes will be cut in the future.
The tax cut will take effect $10$ years from now and the tax rate will drop from $.35$ to $.2$.
The tax cut is permanent.

To value the price of capital, use propositions 4.1 and 4.2.
I'm going to skip to part (iii.), which is different from Professor House's proposed solutions.
But this way is quicker.

The hardest part about this question is timing.
The sequence of tax rates faced by firms is
\begin{align*}
\left\{ \underbrace{\tau_{t+1},\tau_{t+2},\dots,\tau_{t+9}}_{\text{9 of them}},\tau_{t+10},\dots \right\} =
\left\{ \underbrace{.35, .35, \dots, .35}_{\text{9 of them}},.2, .2,\dots \right\}.
\end{align*}

Using equation \eqref{eq:5}, the value of a unit of capital is
\begin{align*}
p_{t} &= \frac{1}{1+r} \Bigg\{ (1-\tau_{t+1})m + \left( \frac{1-\delta}{1+r} \right)(1-\tau_{t+2})m + \cdots  +
\left( \frac{1-\delta}{1+r} \right)^{8} (1-\tau_{t+9}) m \\
&\quad + \left( \frac{1-\delta}{1+r} \right)^{9} (1-\tau_{t+10}) m + \cdots \Bigg\} \\
&= \frac{1}{1+r} (1-\tau_{1})m \Bigg\{ 1 + \left( \frac{1-\delta}{1+r} \right) + \cdots  +
\left( \frac{1-\delta}{1+r} \right)^{8} \Bigg\} \\
&\quad + \frac{1}{1+r} \left( \frac{1-\delta}{1+r} \right)^{9}(1-\tau_{2})m
\Bigg\{
1 + \left( \frac{1-\delta}{1+r} \right) + \left( \frac{1-\delta}{1+r} \right)^{2} + \left( \frac{1-\delta}{1+r} \right)^{3} + \cdots
\Bigg\},
\end{align*}
where $\tau_{1} = .35$ and $\tau_{2} = .20$.
Proposition 4.2 provides a formula for the sum contained in the first curly bracket.
Proposition 4.1 provides a formula for the sum contained in the second curly bracket.
Using those formula:
\begin{align*}
p_{t} &= \frac{m}{1+r} (1-\tau_{1}) \frac{1 - \left( \frac{1-\delta}{1+r} \right)^{9}}{1 - \frac{1-\delta}{1+r}} +
\frac{1}{1+r}\left( \frac{1-\delta}{1+r} \right)^{9}(1-\tau_{2})m \frac{1}{1 - \frac{1-\delta}{1+r}} \\
&= \frac{m}{1+r} (1-\tau_{1}) \frac{1 - \left( \frac{1-\delta}{1+r} \right)^{9}}{1 - \frac{1-\delta}{1+r}} +
\frac{1}{1+r} \left( \frac{1-\delta}{1+r} \right)^{9}(1-\tau_{2})m \frac{1+r}{r + \delta} \\
&= m (1-\tau_{1}) \frac{1 - \left( \frac{1-\delta}{1+r} \right)^{9}}{r+\delta} +
\left( \frac{1-\delta}{1+r} \right)^{9}(1-\tau_{2})m \frac{1}{r + \delta},
\end{align*}
where the last two lines multiply by one in the form of $(1+r) / (1+r)$.

Plugging in values for the two cases leads to the values:
\begin{itemize}
\item $p^{I}_{t} = 7.1094$ with $\delta = .05$;
\item $p^{II}_{t} = 2.1909$ with $\delta = .25$.
\end{itemize}

Now you can answer part (iii.) of the question:
\begin{itemize}
\item $p_{t}^{I}$ moves from $6.5$ [from part c)] to $7.1094$, a 9.38 percent increase; and
\item $p_{t}^{II}$ moves from $2.1667$ [from part c)] to $2.1909$, a 1.12 percent increase.
\end{itemize}

(You may have calculated the increasing from the period-$t+1$ prices.
This is only a matter a timing.)

% \bibliography{c:/Users/rwr/bibliography/bibliography}
% \bibliographystyle{c:/Users/rwr/bibliography/bostonfed}
\end{document}

%%% Local Variables: 
%%% mode: latex
%%% TeX-engine: xetex
%%% End: