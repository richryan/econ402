% Created 2017-12-15 Fri 09:25
\documentclass[presentation]{beamer}
\usepackage{fixltx2e}
\usepackage{graphicx}
\usepackage{longtable}
\usepackage{float}
\usepackage{wrapfig}
\usepackage{rotating}
\usepackage[normalem]{ulem}
\usepackage{amsmath}
\usepackage{textcomp}
\usepackage{marvosym}
\usepackage{wasysym}
\usepackage{amssymb}
% \usepackage{color}
\usepackage{listings}
\usepackage{xfrac}
\usepackage{booktabs}
\usepackage{threeparttable}
\usepackage{siunitx}
\usepackage{tgpagella}
\usepackage[T1]{fontenc}
\usefonttheme{serif}
\setbeamercolor{frametitle}{fg=black}
\setbeamercolor{section in sidebar}{fg=black}
\setbeamercolor{titlelike}{fg=black}
\setbeamercolor{title}{fg=black}
\setbeamercolor{section in toc}{fg=black}
\setbeamercolor{subsection in toc}{fg=black}
\setbeamercolor{block title}{fg=black}
\setbeamercolor{alerted text}{fg=blue}
\setbeamertemplate{section in toc}[sections numbered]
\setbeamertemplate{subsection in toc}[subsections numbered]
\setbeamertemplate{subsubsection in toc}[subsubsections numbered]
\setbeamercolor*{item}{fg=black}
\setbeamertemplate{itemize items}[circle]
\usetheme{default}
\addtobeamertemplate{navigation symbols}{}{%
\usebeamerfont{footline}%
\usebeamercolor[fg]{footline}%
\hspace{1em}%
\insertframenumber/\inserttotalframenumber
}
% \setbeamerfont{footline}{gray}
\usetheme{default}
\author{Rich Ryan \\ richryan@umich.edu}
\date{23 February 2018}
\title{Thinking about \\ fundamentals of asset pricing}
\hypersetup{
  pdfkeywords={},
  pdfsubject={},
  pdfcreator={Emacs 24.5.1 (Org mode 8.2.10)}}
\begin{document}

\maketitle

\begin{frame}
% \frametitle{A very long chain of logic}
\begin{figure}[htbp]
\centerline{\includegraphics[]{mice.pdf}}
\end{figure}
\end{frame}

\begin{frame}
\frametitle{Remember:}
\begin{figure}
\centering
\begin{minipage}{.5\textwidth}
  \centering
  \includegraphics[width=\textwidth]{fig_utility_vs_marginal_utility.eps}
\end{minipage}%
\begin{minipage}{.5\textwidth}
  \centering
  \includegraphics[width=\textwidth]{fig_utility_vs_marginal_utility_2.eps}
\end{minipage}
\end{figure}
\begin{itemize}
\item When consumption is high, marginal utility is low; and
\item when consumption is low, marginal utility is high
\item Assets have value when they pay off in high-MU states
\item E.g.: car/fire insurance
\end{itemize}
\end{frame}

\begin{frame}
\frametitle{Exotic dimensions}

What are the units of
\begin{align*}
\frac{\beta u^{\prime}(c_{t+1})}{u^{\prime}(c_{t})}?
\end{align*}
\begin{itemize}
\item Let $\left[ \cdot \right]$ be the symbol for units
\item $\left[ u^{\prime} \right] \approx \left[ \frac{\Delta \text{utils}}{\Delta c} \right] = \frac{\text{utils}}{\text{bananas}}$
\item $\beta = \frac{\text{utils}_{t}}{\text{utils}_{t+1}}$
\end{itemize}
\end{frame}

\begin{frame}
\frametitle{Exotic dimensions}
Therefore
\begin{align*}
\left[ \frac{\beta u^{\prime}(c_{t+1})}{u^{\prime}(c_{t})} \right] &= \frac{\beta \frac{\text{utils}_{t+1}}{\text{ba}_{t+1}}}{\frac{\text{utils}_{t}}{\text{ba$_{t}$}}} \\
&= \frac{\frac{\text{utils}_{t}}{\text{ba}_{t+1}}}{\frac{\text{utils}_{t}}{\text{ba$_{t}$}}} \\
&= \frac{\text{ba$_{t}$}}{\text{ba}_{t+1}},
\end{align*}
in other words: The stochastic discount factor converts bananas from the future into today's bananas---so you can add bananas to bananas.
\end{frame}


\begin{frame}[label=sec-4-2]{Asset prices in an endowment economy}
Euler equation with price and dividend in terms of banans:
\begin{align*}
p_{t} = \beta \mathbb{E} \left[ \frac{u^{\prime}(c_{t+1})}{u^{\prime}(c_{t})} (y_{t+1}+p_{t+1}) \right].
\end{align*}
Asset-pricing equation:
\begin{align*}
1 = \mathbb{E} \left[ R_{t+1} \frac{\beta u^{\prime}(c_{t+1})}{c^{\prime}(c_{t})} \right],
\end{align*}
where $R_{t+1} = (y_{t+1} + p_{t+1})/p_{t}$.
\alert{Alert}: The return $R_{t+1}$ is inversely related to the price, $p_{t}$.
\end{frame}

\begin{frame}[label=sec-4-3]{Asset prices in an endowment economy}
Expanding the expectation: 
\begin{align*}
1 = \mathbb{E} \left[ R_{t+1} \right] \mathbb{E} \left[ \frac{\beta u^{\prime}(c_{t+1})}{u^{\prime}(c_{t})} \right] + \mathbb{C} \text{ov} \left[ R_{t+1}, \frac{\beta u^{\prime}(c_{t+1})}{u^{\prime}(c_{t})} \right].
\end{align*}
Solving for $R_{t+1}$ yields
\begin{align*}
\mathbb{E}_{t} \left[ R_{t+1} \right] = \frac{1 - \mathbb{C} \text{ov} \left[ R_{t+1}, \frac{\beta u^{\prime}(c_{t+1})}{u^{\prime}(c_{t})} \right]}{ \mathbb{E}_{t} \left[ \frac{\beta u^{\prime}(c_{t+1})}{u^{\prime}(c_{t})} \right]}
\end{align*}
\end{frame}
\begin{frame}[label=sec-4-4]{Asset prices in an endowment economy}
\begin{align*}
\mathbb{E}_{t} \left[ R_{t+1} \right] &= \frac{1 - \mathbb{C} \text{ov} \left[ R_{t+1}, \frac{\beta u^{\prime}(c_{t+1})}{u^{\prime}(c_{t})} \right]}{ \mathbb{E}_{t} \left[ \frac{\beta u^{\prime}(c_{t+1})}{u^{\prime}(c_{t})} \right]} \\
&= R^{f}_{t} \left( 1 - \mathbb{C} \text{ov} \left[ R_{t+1}, \frac{\beta u^{\prime}(c_{t+1})}{u^{\prime}(c_{t})} \right] \right)
\end{align*}
where
\begin{align*}
R^{f}_{t} = \mathbb{E}_{t} \left[ \frac{\beta u^{\prime}(c_{t+1})}{u^{\prime}(c_{t})} \right]^{-1} 
\end{align*}
\end{frame}

\begin{frame}
\frametitle{The riskfree rate of return}
This is the risk-free rate of return in the endowment economy:
\begin{align*}
R^{f}_{t} &= \mathbb{E}_{t} \left[ \frac{\beta u^{\prime}(c_{t+1})}{u^{\prime}(c_{t})} \right]^{-1} \\
\iff 1 &= \mathbb{E}_{t} \left[ R^{f}_{t} \frac{\beta u^{\prime}(c_{t+1})}{u^{\prime}(c_{t})} \right] \\
&= \mathbb{E}_{t} \left[ \frac{1 + 0}{p_{t}} \frac{\beta u^{\prime}(c_{t+1})}{u^{\prime}(c_{t})} \right].
\end{align*}
In other words, the asset pays off $1$ the following period and that's it---no $p_{t+1}$ because there is nothing to sell.
The $R^{f}_{t}$ can come out of the expectation because there is no mystery about $p_{t+1}$ as there is \emph{no} $p_{t+1}$.
\end{frame}

\begin{frame}[label=sec-4-5]{Interpreting asset prices in an endowment economy}
\begin{align*}
\mathbb{E}_{t} \left[ R_{t+1} \right] = R^{f}_{t} \left( 1 - \mathbb{C} \text{ov} \left[ R_{t+1}, \frac{\beta u^{\prime}(c_{t+1})}{u^{\prime}(c_{t})} \right] \right)
\end{align*}

\begin{itemize}
\item ``The expected return on any asset is the riskless return times an adjustment factor for the covariance of the return with the SDF'' (Campbell/2017, p 94)
\item Sps consumption is growing [$u^{\prime}(c_{t+1})$ falling] and the asset returns are high (neg cov)
\item Valuable assets pay off in bad states (car insurace) and this isn't that
\item Drives down demand (price), making the return higher
\end{itemize}
\end{frame}

\begin{frame}
\frametitle{Remember:}
\begin{figure}
\centering
\begin{minipage}{.5\textwidth}
  \centering
  \includegraphics[width=\textwidth]{fig_utility_vs_marginal_utility.eps}
\end{minipage}%
\begin{minipage}{.5\textwidth}
  \centering
  \includegraphics[width=\textwidth]{fig_utility_vs_marginal_utility_2.eps}
\end{minipage}
\end{figure}
\begin{itemize}
\item When consumption is high, marginal utility is low; and
\item when consumption is low, marginal utility is high
\end{itemize}
\end{frame}

\begin{frame}[label=sec-4-6]{Interpreting asset prices in an endowment economy\dots{} \\ \dots{} again}
\begin{align*}
\mathbb{E}_{t} \left[ R_{t+1} \right] = R^{f}_{t} \left( 1 - \mathbb{C} \text{ov} \left[ R_{t+1}, \frac{\beta u^{\prime}(c_{t+1})}{u^{\prime}(c_{t})} \right] \right)
\end{align*}

\begin{itemize}
\item When the covariance term is high, the asset pays off in high marginal-utility states, which is desirable
\item Drives up the price and down the return
\end{itemize}
\end{frame}

\begin{frame}
\frametitle{Same as what House has}
Indeed:
\begin{align*}
\mathbb{E}_{t} \left[ R_{t+1} \right] &=
\frac{1 - \mathbb{C} \text{ov} \left[ R_{t+1}, \frac{\beta u^{\prime}(c_{t+1})}{u^{\prime}(c_{t})} \right]}{
  \mathbb{E}_{t} \left[ \frac{\beta u^{\prime}(c_{t+1})}{u^{\prime}(c_{t})} \right]} \\
\mathbb{E}_{t} \left[ R_{t+1} \right] - \frac{1}{\mathbb{E}_{t} \left[ \frac{\beta u^{\prime}(c_{t+1})}{u^{\prime}(c_{t})} \right]} &=
\frac{- \mathbb{C} \text{ov} \left[ R_{t+1}, \frac{\beta u^{\prime}(c_{t+1})}{u^{\prime}(c_{t})} \right]}{
  \mathbb{E}_{t} \left[ \frac{\beta u^{\prime}(c_{t+1})}{u^{\prime}(c_{t})} \right]} \\
\mathbb{E}_{t} \left[ R_{t+1} \right] - (1+r_{t}^{f}) &= \frac{- \mathbb{C} \text{ov} \left[ R_{t+1}, \frac{\beta u^{\prime}(c_{t+1})}{u^{\prime}(c_{t})} \right]}{
  \mathbb{E}_{t} \left[ \frac{\beta u^{\prime}(c_{t+1})}{u^{\prime}(c_{t})} \right]},
\end{align*}
which is the expression from \texttt{Lecture 9 Handouts}.
\end{frame}

\end{document}