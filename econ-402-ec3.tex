% =====================================================================
% Program: econ-402-ec2.tex
% Purpose: Suggested solutions to the second extra credit.
%
% E-mail: RichRyan@umich.edu
% =====================================================================
\documentclass[12pt]{pracjourn_rwr}
% \documentclass[12pt]{article}

\usepackage[dvipsnames]{xcolor}
\usepackage{url}
\urlstyle{same}
\usepackage{booktabs}
\usepackage{graphicx}
\usepackage{amsmath,amsthm,amssymb}
\usepackage{amsxtra}
\usepackage{natbib}
\usepackage[left=1in,right=1in,top=1in,bottom=1in]{geometry}
\usepackage{caption}
\usepackage{subcaption}
\usepackage{alltt}
\input{ee.sty}
\usepackage{listings}
% \usepackage[inline]{showlabels}
\usepackage{paralist} % inline lists (enumeration)
\usepackage{mathabx} % widebar

% Set new counter to start everything off with a "2" for section 2
\newcounter{rich}
\setcounter{rich}{3}
\newcounter{ryan}
\setcounter{ryan}{5}
\renewcommand{\theryan}{\Alph{ryan}\arabic{rich}}

% Change figure numbering
\usepackage{chngcntr}
\counterwithin{figure}{ryan}

% \newtheorem{theorem}{Theorem}[chapter]
% \newtheorem{lemma}{Lemma}[chapter]
\newtheorem{theorem}{Theorem}[rich]
\newtheorem{lemma}{Lemma}[rich]

% \newtheoremstyle{definition}% Name
% {3pt}% Space above
% {3pt}% Space below
% {}% Body font
% {}% Indent amount
% {\bfseries}% Theorem head font
% {:}% Punctuation after theorem head
% {.5em}% Space after theorem head
% {}% Theorem head spec (can be left empty, meaning "normal")

% \setcounter{chapter}{2}
\theoremstyle{definition}
\newtheorem{definition}{Definition}[rich]
\newtheorem{example}{Example}[rich]
\newtheorem{exercise}{Exercise}[rich]
\newtheorem{prop}{Proposition}[rich]
\newtheorem{assumption}{Assumption}[rich]
\newtheorem{fact}{Fact}[rich]

\theoremstyle{remark}
\newtheorem{remark}{Remark}[ryan]
% \newtheorem{remark}{Remark}[chapter]
% \newtheorem{fact}{Fact}[chapter]

% === Change to homework numbering ===
\renewcommand{\thesection}{E\arabic{rich}.\arabic{section}}

\title{Suggested solution for the third  Extra Credit}
\author{Rich Ryan}
\email{RichRyan@umich.edu}
%% \TPJrevision{2013}{08}{03} % The editors ask you to update this for each %% revision submitted
% \TPrevision{3}{1}{2015}
\license{\texttt{econ-402-ec3.tex}} % May be used if you want a copyright notice at the bottom right of the first page
\abstract{There are two parts to this note.
  In tbe first part, R\&D is added to a baseline Solow economy.
  The rate of growth of capital and output is determined along the balanced growth path of the economy.
  In the second part, a linear IS--LM system is considered.
}
\newinfo\datestart{Date Started}
\newinfo\daterevised{Date Revised} % See the pracjourn class for how to do this
\datestart{April 13, 2018}
\daterevised{April 14, 2018}


% === Spacing in the table of contents
\usepackage{tocloft}
\renewcommand{\cftsecnumwidth}{3em}
\renewcommand{\cftsubsecnumwidth}{5em}
\renewcommand{\cftsecleader}{\cftdotfill{\cftdotsep}}
% Set TOC depth
\setcounter{tocdepth}{2}


\begin{document}
\maketitle

\tableofcontents

\section{R\&D in the Solow growth model}
\label{sec:rd-solow-growth}

In the basic Solow model $A_{t+1} = (1+g_{A})A_{t}$,
where $g_{A} = 0$ in the simplest case or $g_{A} > 0$,
assuming that technological progress occurs at an exogenous rate.
Section \ref{sec:rd-solow-growth} makes technological progress endogenous
by devoting labor resources to research and development.

The economy is populated by $N_{t}$ people who supply labor inelastically.
The economy allocates labor to either work in factories or in the R\&D sector.
Let $n_{t}^{f}$ denote the number of people working in factories and
let $n_{t}^{\text{R\&D}}$ denote the number of people working in the R\&D sector.
Then $N_{t} = n_{t}^{f} + n_{t}^{\text{R\&D}}$.

The rest of the economy is standard; viz.,
the household saves a constant fraction of its income.
The following system of equations summarizes the dynamics that govern the economy:
\begin{align}
\label{eq:1}
Y_{t} &= A_{t}^{1-\alpha} K_{t}^{\alpha} \left( n_{t}^{f} \right)^{1-\alpha} \\
\label{eq:2}
K_{t+1} &= (1-\delta)K_{t} + I_{t} \\
\label{eq:3}
I_{t} &= S_{t} = sY_{t} \\
\label{eq:4}
n_{t}^{\text{R\&D}} &= zN_{t} \\
\label{eq:5}
n_{t}^{f} &= (1-z)N_{t} \\
\label{eq:6}
A_{t+1} &= A_{t} \left( 1+\phi n_{t}^{\text{R\&D}} \right), \quad \phi > 0 \\
\label{eq:7}
N_{t} &= n_{t}^{f} + n_{t}^{\text{R\&D}}. 
\end{align}
We will use the normalization $N_{t} = 1$ and maintain that their is no population growth.
Equations \eqref{eq:4} and \eqref{eq:5} make it possible to solve the economy---the economy devotes a constant fraction of its labor force to both sectors.\footnote{One can imagine extending the model so that labor allocation across sectors is a forward-looking choice.}

\subsection{Part a) Balanced growth}
\label{sec:part-a}

Balanced growth is a list of sequences,
such that quantities are growing at a constant; i.e.,
$\left\{
\left\{ K_{t} \right\}_{t=\tau}^{\infty},
\left\{ Y_{t} \right\}_{t=\tau}^{\infty},
\left\{ C_{t} \right\}_{t=\tau}^{\infty},
\left\{ I_{t} \right\}_{t=\tau}^{\infty} \right\}$
are growing at a constant rate;
prices,
$\left\{ \left\{ w_{t} \right\}_{t=\tau}^{\infty}, \left\{ R_{t} \right\}_{t=\tau}^{\infty} \right\}$,
are growing at constant rate; and
state variables,
$\left\{ \left\{ A_{t} \right\}_{t=\tau}^{\infty},
\left\{ N_{t} \right\}_{t=\tau}^{\infty} \right\}$,
are growing at a constant rate.

Date $\tau$ is an arbitrary start date.
Importantly, all variables can grow at different rates---but all the growth rates must be constant.\footnote{See definition 6.3.}

\subsection{Part b) Solving for the balanced growth}

Example 6.6 goes through a more straightforward calculation.

What's the growth rate of capital along the balanced growth path?
A balanced growth path, from section \ref{sec:part-a}, means
\begin{equation}
\label{eq:8}
\frac{K_{t+1}}{K_{t}} = \frac{K_{t+2}}{K_{t+1}},
\end{equation}
which says that the growth rate of capital between period $t$ and $t+1$ is the same as
the growth rate of capital between period $t+1$ and $t+2$.

Plugging in for $K_{t+1} / K_{t}$ yields
\begin{align*}
\frac{K_{t+1}}{K_{t}} &= \frac{(1-\delta)K_{t} + I_{t}}{K_{t}} = \frac{(1-\delta)K_{t} + sY_{t}}{K_{t}},
\end{align*}
where the last equality uses the Solow assumption that the economy saves a constant fraction of its output,
which is given in \eqref{eq:3}.
Use \eqref{eq:1} in our developing equation:
\begin{align*}
\frac{K_{t+1}}{K_{t}} &=
\frac{(1-\delta)K_{t} + sA_{t}^{1-\alpha}K_{t}^{\alpha} \left( n_{t}^{f} \right)^{1-\alpha}}{K_{t}} \\
&=
\frac{(1-\delta)K_{t} + sA_{t}^{1-\alpha}K_{t}^{\alpha} \left( 1-z \right)^{1-\alpha}}{K_{t}},
\end{align*}
where the second equality uses \eqref{eq:5} and the fact that $N_{t} = 1$.
Simplifying this expression yields
\begin{equation}
\label{eq:9}
\frac{K_{t+1}}{K_{t}} =
(1-\delta) + sA_{t}^{1-\alpha}K_{t}^{\alpha-1} \left( 1-z \right)^{1-\alpha}.
\end{equation}

The same procedure used to derive \eqref{eq:9} yields a similar expression for $K_{t+2} / K_{t+1}$---just
iterate the expression in \eqref{eq:9} one period forward.
Now compute \eqref{eq:8} as
\begin{align*}
(1-\delta) + sA_{t}^{1-\alpha}K_{t}^{\alpha-1} \left( 1-z \right)^{1-\alpha} &=
(1-\delta) + sA_{t+1}^{1-\alpha}K_{t+1}^{\alpha-1} \left( 1-z \right)^{1-\alpha} \\
\therefore sA_{t}^{1-\alpha}K_{t}^{\alpha-1} \left( 1-z \right)^{1-\alpha} &=
sA_{t+1}^{1-\alpha}K_{t+1}^{\alpha-1} \left( 1-z \right)^{1-\alpha},
\end{align*}
where the second equality cancels the decay rates, $1-\delta$.

Now use the expression for $A_{t+1}$ in \eqref{eq:6}:
\begin{align*}
sA_{t}^{1-\alpha}K_{t}^{\alpha-1} \left( 1-z \right)^{1-\alpha} &=
s \left[ A_{t} \left( 1+\phi n_{t}^{\text{R\&D}} \right) \right]^{1-\alpha}K_{t+1}^{\alpha-1} \left( 1-z \right)^{1-\alpha} \\
\therefore sA_{t}^{1-\alpha}K_{t}^{\alpha-1} \left( 1-z \right)^{1-\alpha} &=
s A_{t}^{1-\alpha} \left( 1+\phi z \right)^{1-\alpha}K_{t+1}^{\alpha-1} \left( 1-z \right)^{1-\alpha},
\end{align*}
where the last line uses \eqref{eq:4} and the fact that $N_{t} = 1$.

Canceling terms in our developing equation yields
\begin{align*}
K_{t}^{\alpha-1} &=
\left( 1+\phi z \right)^{1-\alpha}K_{t+1}^{\alpha-1} \\
\therefore \left( \frac{K_{t}}{K_{t+1}} \right)^{\alpha-1} &= \left( 1+\phi z \right)^{1-\alpha} \\
\therefore \left( \frac{K_{t+1}}{K_{t}} \right)^{1-\alpha} &= \left( 1+\phi z \right)^{1-\alpha} \\
\therefore \left[ \left( \frac{K_{t+1}}{K_{t}} \right)^{1-\alpha} \right]^{\frac{1}{1-\alpha}} &=
\left[ \left( 1+\phi z \right)^{1-\alpha} \right]^{\frac{1}{1-\alpha}}.
\end{align*}
Which simplifies to the lovely expression
\begin{equation}
\label{eq:10}
\frac{K_{t+1}}{K_{t}} = 1 + \phi z.
\end{equation}

\subsection{Part c) Changes in $z$}

Looking at \eqref{eq:10}, an increase in $z$ increases the growth rate of capital.
How does $z$ affect the growth rate of output?
\begin{align*}
\frac{Y_{t+1}}{Y_{t}} &=
\frac{A_{t+1}^{1-\alpha}K_{t+1}^{\alpha} \left( n_{t+1}^{f} \right)^{1-\alpha}}{
  A_{t}^{1-\alpha}K_{t}^{\alpha} \left( n_{t}^{f} \right)^{1-\alpha}} =
\frac{A_{t+1}^{1-\alpha}K_{t+1}^{\alpha} \left( 1-z \right)^{1-\alpha}}{
  A_{t}^{1-\alpha}K_{t}^{\alpha} \left( 1-z \right)^{1-\alpha}} \\
&= \frac{A_{t+1}^{1-\alpha}K_{t+1}^{\alpha}}{A_{t}^{1-\alpha}K_{t}^{\alpha}} \\
&= \left( \frac{A_{t+1}}{A_{t}} \right)^{1-\alpha} \left( \frac{K_{t+1}}{K_{t}} \right)^{\alpha},
\end{align*}
where the derivation uses \eqref{eq:5} and $N_{t}= 1$.

Now use \eqref{eq:6} and \eqref{eq:10} in our developing equation:
\begin{align*}
\frac{Y_{t+1}}{Y_{t}} &=
\left( \frac{A_{t}\left( 1+\phi z \right)}{A_{t}} \right)^{1-\alpha} \left( 1+\phi z \right)^{\alpha} \\
&= \left( 1+\phi z \right)^{1-\alpha} \left( 1+\phi z \right)^{\alpha} \\
&= \left( 1+\phi z \right),
\end{align*}
which says that output is growing at the same rate as technology along the balanced growth path.
Increasing $z$ increases the growth rate of output along the balanced growth path. 

\section{A linear IS--LM model}

This question is about the difference between a flat SRAS curve (like in Mankiw) and
an upward-sloping SRAS curve (the presentation in class).

The LM curve is given by
\begin{align}
\begin{split}
\label{eq:11}
M - P &= Y - \phi i \\
i &= r + \pi^{E},
\end{split}
\end{align}
where $\pi^{E}$ is expected inflation, which is given exogenously.
The parameter $\phi$ is intereste elasticity of money demand.

The IS curve is given by
\begin{align}
\begin{split}
\label{eq:12}
Y &= C + I + G \\
C &= \overline{C} + \text{MPC} \times Y \\
I &= \overline{I} - \varepsilon r,
\end{split}
\end{align}
where government spending, $G$, is exogenously given.
$0 < \text{MPC} < 1$ is the marginal propensity to consume.
$\varepsilon$ is the interest elasticity of investment demand.

\subsection{Part a) The IS equation}

Combine the equations in \eqref{eq:12} to get an equation explicitly in terms of $Y$ and $r$:
\begin{equation}
\label{eq:13}
Y = C + I + G = \overline{C} + \text{MPC} \times Y + \overline{I} - \varepsilon r + G,
\end{equation}
which is explicitly in terms of $Y$ and $r$; recall that $G$ is exogenous.

Assuming $r$ is unchanged, how does $Y$ depend on $G$?\footnote{Small aside: we derived this relationship in Remark 9.6.
  The derivation in Remark 9.6 used the implicit function theorem (or the chain rule) because we did not know the explicit form of $C(Y,r)$.
  Here we do.
  Which allows us to solve for $Y$ in terms of $G$ \textit{explicitly}.
  We can then directly differentiate the relationship.}
Use \eqref{eq:13} to solve for $Y$:
\begin{align*}
Y &= \overline{C} + \text{MPC} \times Y + \overline{I} - \varepsilon r + G \\
\therefore Y \left( 1+ \text{MPC} \right) &= \overline{C} + \overline{I} - \varepsilon r + G \\
\therefore Y &= \frac{\overline{C} + \overline{I} - \varepsilon r + G}{1-\text{MPC}}.
\end{align*}
How $Y$ depends on $G$ is a straightforward calculation:
\begin{equation}
\label{eq:14}
\frac{d Y}{d G} = \frac{1}{1-\text{MPC}}.
\end{equation}

Equation \eqref{eq:14} looks right, considering the intuition we have from the Keynesian cross.
An increase in $G$ directly increases $Y$, one-for-one, in the first round.
But a higher $Y$ encourages spending through $C(Y,r)$.
In the second round, $C$ and therefore $Y$ increases by $\text{MPC}$.
This increases $Y$ again, which spurs an increase in consumption of $\text{MPC}^{2}$.
This process continues.
The total change in $Y$ is
\begin{align*}
1 + \text{MPC} + \text{MPC}^{2} + \text{MPC}^{3} + \cdots = \frac{1}{1-\text{MPC}}.
\end{align*}

\subsection{Part b) The LM system and re-evaluating $dY / dG$}

Combine the expressions in \eqref{eq:11}:
\begin{align*}
M - P &= Y - \phi i \\
\therefore M - P &= Y - \phi \left( r + \pi^{E} \right) \\
\therefore \phi r &= Y - M + P - \phi \pi^{E} \\
\therefore r &= \frac{1}{\phi} \left( Y - M + P - \phi \pi^{E} \right).
\end{align*}
Note: plotting this relationship in $r$--$Y$ space would give an upward-sloping relationship.
Good, because that's how we've been drawing the LM curve.

Now we want to use the expression for $r$, coming from the LM block,
in the IS block:
\begin{align*}
Y &= \overline{C} + \text{MPC} Y + \overline{I} + G - \varepsilon r \\
\therefore Y &=
\overline{C} + \text{MPC} Y + \overline{I} + G - \frac{\varepsilon}{\phi} \left( Y - M + P - \phi \pi^{E} \right) \\
\therefore Y \left( 1 - \text{MPC} + \frac{\varepsilon}{\phi} \right) &=
\overline{C} + \overline{I} + G - \frac{\varepsilon}{\phi} \left(- M + P - \phi \pi^{E} \right) \\
\therefore Y &= \frac{\overline{C} + \overline{I} + G - \frac{\varepsilon}{\phi} \left(- M + P - \phi \pi^{E} \right)}{1 - \text{MPC} + \frac{\varepsilon}{\phi}}.
\end{align*}

Now
\begin{equation}
\label{eq:15}
\frac{d Y}{d G} = \frac{1}{1 - \text{MPC} + \frac{\varepsilon}{\phi}}.
\end{equation}

\subsection{Part c) Comparing parts a) and b)}

What happens when the LM curve is allowed to affect $dY / dG$?
Comparing \eqref{eq:14} to \eqref{eq:15}, it is true that
\begin{align*}
\frac{1}{1 - \text{MPC} + \frac{\varepsilon}{\phi}} < \frac{1}{1-\text{MPC}},
\end{align*}
which follows from the fact that $1 / (a +x)$ is decreasing in $x$.

What are we talking about?
Look at figure \ref{fig:is-lm}.
When you calculate $d Y/ dG$ as $1/(1-\text{MPC})$,
you're holding $r$ fixed and seeing how much $Y$ changes.
That's shifting the IS curve rightward from IS$_{1}$ to IS$_{2}$ to point $A$.
But that's not the total effect in $Y$---$r$ changes as $Y$ changes.
Insert the LM relationship here derived in Part b).

Doing so yields $d Y/ d G = 1 / ( 1 - \text{MPC} + \varepsilon / \phi)$.
The total effect moves the economy to $B$.

\begin{figure}[htbp]
\centerline{\includegraphics[]{figs/econ402_IS_LM.eps}}
\caption[]{\label{fig:is-lm} Increasing $G$ in the IS--LM framework.}
\end{figure}

The parameters $\varepsilon$ and $\phi$ affect $d Y/ d G$ because these are the shapes of the curves.
(The curves are linear here but you get the idea.)
$d Y / d G$ is decreasing in $\varepsilon$:
\begin{align*}
\frac{\partial}{\partial \varepsilon} \left[ \frac{d Y}{d G} \right] =
\frac{-1}{\left( 1 - \text{MPC} + \frac{\varepsilon}{\phi} \right)^{2}} \frac{1}{\phi} < 0.
\end{align*}
Why is this negative?
Mechanically, looking at \eqref{eq:13},
$-1 / \varepsilon$ is the slope of the IS curve in figure \ref{fig:is-lm}.
Increasing $\varepsilon$ makes $-1/\varepsilon$ flatter.
Imagine rotating IS$_{2}$ around $A$.
The total effect will cause $Y$ to increase by less.

Why?
Note the $C = \overline{C} + \text{MPC} \times Y$ and $C$ does not depend on $r$.
$Y$ only depends on $r$ through $I$ (like in the Mankiw book).
The shift from IS$_{1}$ to IS$_{2}$ causes $r$ to rise.
$I$ depends negatively on $r$, a relationship governed by $\varepsilon$.
The bigger is $\varepsilon$ the more $I$ decreases, pushing down $Y$.

How does $\phi$ affect $d Y / d G$?
\begin{align*}
\frac{\partial}{\partial \phi} \left[ \frac{d Y}{d G} \right] =
\frac{-1}{\left( 1 - \text{MPC} + \frac{\varepsilon}{\phi} \right)^{2}} \frac{-\epsilon}{\phi^{2}} > 0.
\end{align*}
Mechanically, $1 / \phi$ is the slope of the LM curve in figure \ref{fig:is-lm}.
Increasing $\phi$ makes the LM curve flatter.
Figure \ref{fig:is-lm} shows that a flatter LM curve will push out the increase in $Y$.
Intuitively,
the bigger is $\phi$, the less $r$ will respond to the increased pressure on the interest rate,
which causes $I$ to fall by less.
The result is higher $Y$.

% \bibliography{c:/Users/rwr/bibliography/bibliography}
% \bibliographystyle{c:/Users/rwr/bibliography/bostonfed}
\end{document}

%%% Local Variables: 
%%% mode: latex
%%% TeX-engine: xetex
%%% End: