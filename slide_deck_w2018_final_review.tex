% Created 2018-04-17 Tue 14:04
\documentclass[presentation,dvipsnames]{beamer}
% \usepackage[dvipsnames]{xcolor}
% \usepackage{fixltx2e}
% \usepackage{graphicx}
% \usepackage{longtable}
% \usepackage{float}
% \usepackage{wrapfig}
% \usepackage{rotating}
% \usepackage[normalem]{ulem}
\usepackage{amsmath,amssymb,amsthm}
\newtheorem{prop}{Proposition}
% \usepackage{marvosym}
% \usepackage{wasysym}
% \usepackage{amssymb}
\usepackage{enumitem}
\usepackage[T1]{fontenc}
\usepackage{textcomp}           % to get \pounds to work in math mode

\renewcommand{\pounds}{\text{\textsterling}{ }}

% \usepackage[math]{iwona}
% \usepackage{listings}
\usetheme{default}
\author{}
\date{April 19 \& 20, 2018}
\title{W2018 ECON 402 Final Review}
\hypersetup{
  pdfkeywords={},
  pdfsubject={},
  pdfcreator={Emacs 24.5.1 (Org mode 8.2.10)}}
\begin{document}

\maketitle
\begin{frame}{Outline}
% \setcounter{tocdepth}{2}
\tableofcontents
% \tableofcontents[ 
%     currentsubsection, 
%     % hideothersubsections, 
%     sectionstyle=show/hide, 
%     subsectionstyle=show/shaded, 
%     ] 
\end{frame}

\section{Important concepts}

\begin{frame}
\frametitle{Outline}
\tableofcontents[currentsection]
% currentsubsection,
% hideothersubsections,
% sectionstyle=show/hide,
% subsectionstyle=show]
\end{frame}

\subsection{Intertemporal choice}
\label{sec-2}

\begin{frame}[label=sec-2-1]{Intertemporal choice: Environment}
The agent makes decisions about consumption and saving to maximize utility
\begin{align*}
\mathcal{U}(c_{1},c_{2}) = \ln c_{1} + \beta \ln c_{2},
\end{align*}
where $\beta$ is the subjective discount factor and choices are made subject to
\begin{align*}
c_{1} + s_{1} &= y_{1} \\
c_{2} + s_{2} &= y_{2} + (1+r_{1})s_{1}.
\end{align*}
\begin{itemize}[label={--}]
\item $s_{2} = 0$ because there is no period 3
\item Borrow and lend freely ($s_{1}$ can be $<0$, $=0$, or $>0$):
\begin{align*}
s_{1} &= \frac{c_{2}-y_{2}}{1+r_{1}} \\
\therefore c_{1} + \frac{c_{2}}{1+r_{1}} &= y_{1} + \frac{y_{2}}{1+r_{1}}
\end{align*}
\item Intertemporal budget constraint or ``unified'' budget constraint
\end{itemize}
\end{frame}

\begin{frame}[label=sec-2-1]{Intertemporal choice: The choice part}
\centerline{\includegraphics[width=\textwidth]{figs/econ402_figure_2_2.PNG}}
\end{frame}

\begin{frame}[label=sec-2-1]{Intertemporal choice: Math behind the choice part}
\begin{align*}
\mathcal{U} = \ln \left( y_{1} + \frac{y_{2}}{1+r_{1}} - \frac{c_{2}}{1+r_{1}} \right) +
\beta \ln c_{2}
\end{align*}
First-order condition for $c_{2}$:
\begin{align*}
0 &= \frac{1}{c_{1}} \frac{-1}{1+r_{1}} + \beta \frac{1}{c_{2}} \\
\therefore \frac{1}{c_{1}} &= \beta (1+r) \frac{1}{c_{2}}
\end{align*}
In the figure:
\begin{align*}
- \frac{1 / c_{1}}{\beta / c_{2}} &= - (1+r_{1}) \\
\text{MRS} &= \text{price ratio}
\end{align*}
\end{frame}

\begin{frame}
\frametitle{Intertemporal choice: Regions of borrowing and lending}
\centerline{\includegraphics[width=\textwidth]{figs/econ402_figure_2_3.PNG}}
\end{frame}

\begin{frame}
\frametitle{Lenders and the interest rate rises}
\begin{prop}
\label{prop:1}
If an agent is a lender and the interest rate rises,
then they will remain a lender.
\end{prop}
You can always consumer your endowment:
\centerline{\includegraphics[width=\textwidth]{figs/econ402_figure_2_4.PNG}}
\end{frame}

\begin{frame}
\frametitle{Lenders and the interest rate rises}
\begin{prop}
\label{prop:1}
If an agent is a lender and the interest rate rises,
then they will remain a lender.
\end{prop}

\begin{proof}
\begin{itemize}[label={--}]
\item You can always consumer your endowment (gives rotation)
\item Is it possible that after the increase in the interest rate the agent chooses a point to the right of the endowment?
\item \emph{Absolutely not}. Revealed preference
\item They could have afforded all these points initially
\item Must be to the left of $y_{1} \implies$ remain lender
\end{itemize}
\end{proof}
\end{frame}

\subsection{Consumption Euler equation, generally}
\begin{frame}[label=sec-2-2]{Consumption Euler equation, generally}
Infinitely-lived agent chooses streams of consumption to maximize
\begin{align*}
\mathcal{U}(c_{0},c_{1},\dots) = \sum\limits_{t=0}^{\infty} \beta^{t} u(c_{t})
\end{align*}
subject to the Tom Brady budget constraint:
\begin{align*}
\sum\limits_{t=0}^{\infty} \frac{c_{t}}{(1+r)^{t}} = \sum\limits_{t=0}^{\infty} \frac{y_{t}}{(1+r)^{t}} +
\underbrace{A_{0}}_{\text{D.~Trump Jr.~initial assets}}
\end{align*}
\end{frame}

\begin{frame}[label=sec-2-2]{Consumption Euler equation, generally}
\begin{align*}
\text{MU}(c_{t}) := \frac{\partial \mathcal{U}}{\partial c_{t}} = \beta^{t} u^{\prime}(c_{t})
\end{align*}
Tangency condition (think of the two-period case):
\begin{align*}
\text{MU}(c_{t}) &= (1+r) \text{MU}(c_{t+1}) \\
\therefore \beta^{t} u^{\prime}(c_{t}) &= (1+r) \beta^{t+1}u^{\prime}(c_{t+1}) \\
\therefore u^{\prime}(c_{t}) &= (1+r) \beta u^{\prime}(c_{t+1})
\end{align*}
\begin{itemize}[label={--}]
\item Could be $r_{t}$ but have to deal carefully with the Tom Brady budget constraint
\item If log preferences:
\begin{align*}
\frac{1}{c_{t}} = (1+r)\beta \frac{1}{c_{t+1}}
\end{align*}
\end{itemize}
\end{frame}

\begin{frame}[label=sec-2-2]{Consumption Euler equation, generally}
\framesubtitle{\dots or more generally, if Professor House wants to be funny}
\begin{align*}
u^{\prime}(c_{t}) = \beta (1+r_{1}) \mathbb{E}_{t} \left[ u^{\prime}(c_{t+1}) \right],
\end{align*}
where $\mathbb{E}_{t}$ is the expectations sign,
which takes into account of all information available at time $t$
(hence the $t$ subscript, if that wasn't abundantly clear)
\end{frame}

\begin{frame}[label=sec-2-3]{Sums and partial sums (before we can do PIH)}
\begin{prop}
\label{prop:2}
For $|\alpha| < 1$,
\begin{align*}
1 + \alpha + \alpha^{2} + \alpha^{3} + \cdots = \sum\limits_{t=0}^{\infty} \alpha^{t} = \frac{1}{1-\alpha}.
\end{align*}
\end{prop}
\begin{proof}
We are interested in the sum $S$:
\begin{align*}
S &= 1 + \alpha + \alpha^{2} + \alpha^{3} + \cdots \\
\implies \alpha S &= \alpha + \alpha^{2} + \alpha^{3} + \cdots \\
\implies S - \alpha S &= (1-\alpha)S = 1.
\end{align*}
Thus
\begin{align*}
S = \frac{1}{1-\alpha}.
\end{align*}
\end{proof}
\end{frame}

\begin{frame}[label=sec-2-3]{Sums and partial sums (before we can do PIH)}
\begin{prop}
\label{prop:2}
For $|\alpha| < 1$ and $n \in \mathbb{N}$
\begin{align*}
\sum\limits_{j=0}^{n} \alpha^{j} = 1+\alpha + \alpha^{2} + \cdots + \alpha^{n} = \frac{1-\alpha^{n+1}}{1-\alpha}.
\end{align*}
\end{prop}
\begin{proof}
Use the same steps as above.
\end{proof}
\end{frame}

\subsection{PIH}
\begin{frame}[label=sec-2-4]{Perfect consumption smoothing}
\framesubtitle{With $\beta (1+r) = 1$}
From
\begin{align*}
\frac{1}{c_{t}} &= \beta (1+r) \frac{1}{c_{t+1}} \\
\implies \frac{c_{t+1}}{c_{t}} &= \beta (1+r) \\
&= 1.
\end{align*}
Therefore
\begin{align*}
c_{t+1} = c_{t} = c.
\end{align*}
\end{frame}

\begin{frame}
\frametitle{Use $c_{t+1} = c_{t} = c$ in the lifetime budget constraint}
Use our geometric-series result:
\begin{align*}
\sum\limits_{t=0}^{\infty} \frac{c_{t}}{(1+r)^{t}} &=
\sum\limits_{t=0}^{\infty} \frac{y_{t}}{(1+r)^{t}} + A_{0} \\
\therefore \sum\limits_{t=0}^{\infty} \frac{c}{(1+r)^{t}} &=
\sum\limits_{t=0}^{\infty} \frac{y_{t}}{(1+r)^{t}} + A_{0} \\
\therefore \frac{c}{1- \frac{1}{1+r}} &=
\sum\limits_{t=0}^{\infty} \frac{y_{t}}{(1+r)^{t}} + A_{0} \\
\therefore \frac{c}{\frac{r}{1+r}} &=
\sum\limits_{t=0}^{\infty} \frac{y_{t}}{(1+r)^{t}} + A_{0}.
\end{align*}
Thus
\begin{align*}
c = \frac{r}{1+r} \left[ \sum\limits_{t=0}^{\infty} \frac{y_{t}}{(1+r)^{t}} + A_{0} \right].
\end{align*}
\end{frame}

\begin{frame}
\frametitle{Perfect consumption smoothing}
\begin{align*}
c = \frac{r}{1+r} \left[ \sum\limits_{t=0}^{\infty} \frac{y_{t}}{(1+r)^{t}} + A_{0} \right]
\end{align*}
If $y_{t} = y$, then
\begin{align*}
c &= \frac{r}{1+r} \left[ \sum\limits_{t=0}^{\infty} \frac{y}{(1+r)^{t}} + A_{0} \right] \\
&= \frac{r}{1+r} \left[ \frac{y}{\frac{r}{1+r}} + A_{0} \right] \\
&= y + \frac{r}{1+r} A_{0}.
\end{align*}

Strong motive to smooth consumption.
\end{frame}

\begin{frame}
\frametitle{Strong motive to smooth consumption}
Ben Bernanke: That can't be good:
\centerline{\includegraphics[width=\textwidth]{figs/econ402_FRED_real_personal_consumption_expenditures.png}}
\end{frame}

\begin{frame}
\frametitle{Transitory change in income}
\framesubtitle{The barkeep and the zoo owner (???)}
Consider the income stream hit by a \textcolor{Plum}{transitory} shock
\begin{align*}
\left\{ y_{t} \right\}_{t=0}^{\infty} = \left\{ y_{0},y_{1},y_{2}+\spadesuit,y_{3},y_{4},\dots \right\}.
\end{align*}

Then
\begin{align*}
c = \frac{r}{1+r} \left[ A_{0} + \sum\limits_{t=0}^{\infty} \frac{y_{t}}{(1+r)^{t}} + \frac{\spadesuit}{(1+r)^{2}} \right]
\end{align*}
and
\begin{align*}
\frac{\partial c}{\partial \spadesuit} = \frac{1}{1+r} \frac{1}{(1+r)^{2}} \approx r
\end{align*}
\end{frame}

\begin{frame}
\frametitle{Permanent change in income}
\framesubtitle{The barkeep and the zoo owner again (???)}
Consider the income stream hit by a \textcolor{Plum}{permanent}
\begin{align*}
\left\{ y_{t} \right\}_{t=0}^{\infty} = \left\{ y_{0} + \clubsuit,y_{1}+\clubsuit,y_{2}+\clubsuit,y_{3}+\clubsuit,y_{4}+
\clubsuit,\dots \right\}.
\end{align*}

Then
\begin{align*}
c &= \frac{r}{1+r} \left[ A_{0} + \sum\limits_{t=0}^{\infty} \frac{y_{t}+\clubsuit}{(1+r)^{t}} \right] \\
&= \frac{r}{1+r} \left[ A_{0} + \sum\limits_{t=0}^{\infty} \frac{y_{t}}{(1+r)^{t}}  + \clubsuit \frac{1+r}{r}\right]
\end{align*}
and
\begin{align*}
\frac{\partial c}{\partial \clubsuit} = 1
\end{align*}

$\implies$ Smooth out consumption (for log preferences)
\end{frame}

\begin{frame}
\frametitle{PIH}
Temporary change in income:
\begin{align*}
\frac{\partial c}{\partial y} = \text{MPC} \approx r
\end{align*}

Permanent change in income:
\begin{align*}
\frac{\partial c}{\partial y} = \text{MPC} = 1
\end{align*}
\end{frame}

\begin{frame}
\frametitle{Milton Friedman}
Aggregate vs individual-level data:
\centerline{\includegraphics[width=\textwidth]{rwr_lecture_notes/econ402_w2018_class5_milton_friedman.png}}

$\implies$ fluctuations at individual-level are averaged out in aggregate
\end{frame}

\subsection{Choice under uncertainty}

\begin{frame}
\frametitle{Choice under uncertainty}
\frametitle{Taking stock of the long problem on Midterm Exam 1}
The setup is a two-period endowment economy with uncertainty:
\centerline{\includegraphics[width=\textwidth]{rwr_lecture_notes/econ402_w2018_class5_uncertainty_unfolding.png}}
\begin{itemize}[label={--}]
\item Everyone gets the same endowment ($\implies$ everyone gets the same allocation)
\item Agents make saving choices before the uncertainty is resolved
\end{itemize}
\end{frame}

\begin{frame}
\frametitle{Choice under uncertainty: Consumption Euler equation}
Consumption Euler equation:
\begin{align*}
\text{MU}(c_{1}) &= (1+r_{1}) \mathbb{E} \left[ \text{MU}(c_{2}) \right] \\
\therefore \frac{1}{c_{1}} &= (1+r_{1}) \mathbb{E} \left[ \beta \frac{1}{c_{2}} \right]
\end{align*}

Find that
\begin{align*}
1+r_{1} = \frac{1-\varepsilon^{2}}{\beta}
\end{align*}
\textcolor{RubineRed}{The interest rate is decreasing in $\varepsilon$}

\vspace{1em}

Why?
\end{frame}

\begin{frame}
\frametitle{Choice under uncertainty: Small $\varepsilon$}
\centerline{\includegraphics[width=\textwidth]{rwr_lecture_notes/econ402_w2018_class5_small_epsilon.png}}
\end{frame}

\begin{frame}
\frametitle{Choice under uncertainty: Large $\varepsilon$}
\centerline{\includegraphics[width=\textwidth]{rwr_lecture_notes/econ402_w2018_class5_large_epsilon.png}}
\end{frame}

\begin{frame}
\frametitle{Choice under uncertainty: Story}
\begin{itemize}[label={--}]
\item Remember this is one of those endowment economies with $S_{1} = 0$
\item Agents \emph{trying} to save more with bigger $\varepsilon$ (using story from the figure)
\item How do you prevent agents from saving?
\item Lower $r_{1}$
\item $r_{1}$ must go down to prevent saving
\item (Nefarious Professor House: profit function is convex)
\end{itemize}
\end{frame}

\begin{frame}
\frametitle{Profit function is convex}
\centerline{\includegraphics[width=\textwidth]{../econ605/figs/profit-convex.eps}}
\end{frame}

\subsection{Fundamentals of asset pricing}

\begin{frame}
\frametitle{Fundamentals of asset pricing: \texttt{Lecture 9 Handouts}}
\begin{align*}
\mathbb{E}_{t} \left[ R_{t+1} \right] &=
\frac{1 - \mathbb{C} \text{ov} \left[ R_{t+1}, \frac{\beta u^{\prime}(c_{t+1})}{u^{\prime}(c_{t})} \right]}{
  \mathbb{E}_{t} \left[ \frac{\beta u^{\prime}(c_{t+1})}{u^{\prime}(c_{t})} \right]} \\
\mathbb{E}_{t} \left[ R_{t+1} \right] - \frac{1}{\mathbb{E}_{t} \left[ \frac{\beta u^{\prime}(c_{t+1})}{u^{\prime}(c_{t})} \right]} &=
\frac{- \mathbb{C} \text{ov} \left[ R_{t+1}, \frac{\beta u^{\prime}(c_{t+1})}{u^{\prime}(c_{t})} \right]}{
  \mathbb{E}_{t} \left[ \frac{\beta u^{\prime}(c_{t+1})}{u^{\prime}(c_{t})} \right]} \\
\mathbb{E}_{t} \left[ R_{t+1} \right] - (1+r_{t}^{f}) &= \frac{- \mathbb{C} \text{ov} \left[ R_{t+1}, \frac{\beta u^{\prime}(c_{t+1})}{u^{\prime}(c_{t})} \right]}{
  \mathbb{E}_{t} \left[ \frac{\beta u^{\prime}(c_{t+1})}{u^{\prime}(c_{t})} \right]},
\end{align*}
which is an expression from \texttt{Lecture 9 Handouts}
\end{frame}

\begin{frame}
\frametitle{Fundamentals of asset pricing: \texttt{Lecture 9 Handouts}}
\centerline{\includegraphics[]{figs/dr_evil.jpg}}
\end{frame}

\begin{frame}[label=sec-4-5]{Interpreting asset prices in an endowment economy}
Same as:
\begin{align*}
\mathbb{E}_{t} \left[ R_{t+1} \right] = R^{f}_{t} \left( 1 - \mathbb{C} \text{ov} \left[ R_{t+1}, \frac{\beta u^{\prime}(c_{t+1})}{u^{\prime}(c_{t})} \right] \right)
\end{align*}

\begin{itemize}[label={--}]
\item The expected return on any asset is the riskless return times an adjustment factor for the covariance of the return with the SDF
\item Sps consumption is growing [$u^{\prime}(c_{t+1})$ falling] and the asset returns are high (neg cov)
\item Valuable assets pay off in bad states (car insurace) and this isn't that
\item Drives down demand (price), making the return higher
\end{itemize}
\end{frame}

\begin{frame}
\frametitle{Remember:}
\begin{figure}
\centering
\begin{minipage}{.5\textwidth}
  \centering
  \includegraphics[width=\textwidth]{fig_utility_vs_marginal_utility.eps}
\end{minipage}%
\begin{minipage}{.5\textwidth}
  \centering
  \includegraphics[width=\textwidth]{fig_utility_vs_marginal_utility_2.eps}
\end{minipage}
\end{figure}
\begin{itemize}[label={--}]
\item When consumption is high, marginal utility is low; and
\item when consumption is low, marginal utility is high
\item Assets have value when they pay off in high-MU states
\item E.g.: car/fire insurance
\end{itemize}
\end{frame}

% \begin{frame}
% \frametitle{Exotic dimensions}

% What are the units of
% \begin{align*}
% \frac{\beta u^{\prime}(c_{t+1})}{u^{\prime}(c_{t})}?
% \end{align*}
% \begin{itemize}[label={--}]
% \item Let $\left[ \cdot \right]$ be the symbol for units
% \item $\left[ u^{\prime} \right] \approx \left[ \frac{\Delta \text{utils}}{\Delta c} \right] = \frac{\text{utils}}{\text{bananas}}$
% \item $\beta = \frac{\text{utils}_{t}}{\text{utils}_{t+1}}$
% \end{itemize}
% \end{frame}

% \begin{frame}
% \frametitle{Exotic dimensions}
% Therefore
% \begin{align*}
% \left[ \frac{\beta u^{\prime}(c_{t+1})}{u^{\prime}(c_{t})} \right] &= \frac{\beta \frac{\text{utils}_{t+1}}{\text{ba}_{t+1}}}{\frac{\text{utils}_{t}}{\text{ba$_{t}$}}} \\
% &= \frac{\frac{\text{utils}_{t}}{\text{ba}_{t+1}}}{\frac{\text{utils}_{t}}{\text{ba$_{t}$}}} \\
% &= \frac{\text{ba$_{t}$}}{\text{ba}_{t+1}},
% \end{align*}
% in other words: The stochastic discount factor converts bananas from the future into today's bananas---so you can add bananas to bananas.
% \end{frame}

\begin{frame}[label=sec-4-2]{Returns are inversely related to prices}
Euler equation: % with price and dividend in terms of banans:
\begin{align*}
p_{t} = \beta \mathbb{E} \left[ \frac{u^{\prime}(c_{t+1})}{u^{\prime}(c_{t})} (y_{t+1}+p_{t+1}) \right].
\end{align*}
Asset-pricing equation:
\begin{align*}
1 = \mathbb{E} \left[ R_{t+1} \frac{\beta u^{\prime}(c_{t+1})}{c^{\prime}(c_{t})} \right],
\end{align*}
where $R_{t+1} = (y_{t+1} + p_{t+1})/p_{t}$.
\alert{Alert}: The return $R_{t+1}$ is inversely related to the price, $p_{t}$.

\vspace{1em}

\textcolor{RubineRed}{Story about prices might be easier to interpret!}
\end{frame}

\begin{frame}[label=sec-4-3]{Asset prices in an endowment economy}
Expanding the expectation: 
\begin{align*}
1 = \mathbb{E} \left[ R_{t+1} \right] \mathbb{E} \left[ \frac{\beta u^{\prime}(c_{t+1})}{u^{\prime}(c_{t})} \right] + \mathbb{C} \text{ov} \left[ R_{t+1}, \frac{\beta u^{\prime}(c_{t+1})}{u^{\prime}(c_{t})} \right].
\end{align*}
Solving for $R_{t+1}$ yields
\begin{align*}
\mathbb{E}_{t} \left[ R_{t+1} \right] = \frac{1 - \mathbb{C} \text{ov} \left[ R_{t+1}, \frac{\beta u^{\prime}(c_{t+1})}{u^{\prime}(c_{t})} \right]}{ \mathbb{E}_{t} \left[ \frac{\beta u^{\prime}(c_{t+1})}{u^{\prime}(c_{t})} \right]}
\end{align*}
\end{frame}

\begin{frame}[label=sec-4-4]{Asset prices in an endowment economy}
\begin{align*}
\mathbb{E}_{t} \left[ R_{t+1} \right] &= \frac{1 - \mathbb{C} \text{ov} \left[ R_{t+1}, \frac{\beta u^{\prime}(c_{t+1})}{u^{\prime}(c_{t})} \right]}{ \mathbb{E}_{t} \left[ \frac{\beta u^{\prime}(c_{t+1})}{u^{\prime}(c_{t})} \right]} \\
&= R^{f}_{t} \left( 1 - \mathbb{C} \text{ov} \left[ R_{t+1}, \frac{\beta u^{\prime}(c_{t+1})}{u^{\prime}(c_{t})} \right] \right)
\end{align*}
where
\begin{align*}
R^{f}_{t} = \mathbb{E}_{t} \left[ \frac{\beta u^{\prime}(c_{t+1})}{u^{\prime}(c_{t})} \right]^{-1} 
\end{align*}
\end{frame}

\begin{frame}
\frametitle{The riskfree rate of return}
This is the risk-free rate of return in the endowment economy:
\begin{align*}
R^{f}_{t} &= \mathbb{E}_{t} \left[ \frac{\beta u^{\prime}(c_{t+1})}{u^{\prime}(c_{t})} \right]^{-1} \\
\iff 1 &= \mathbb{E}_{t} \left[ R^{f}_{t} \frac{\beta u^{\prime}(c_{t+1})}{u^{\prime}(c_{t})} \right] \\
&= \mathbb{E}_{t} \left[ \frac{1 + 0}{p_{t}} \frac{\beta u^{\prime}(c_{t+1})}{u^{\prime}(c_{t})} \right].
\end{align*}
In other words, the asset pays off $1$ the following period and that's it---no $p_{t+1}$ because there is nothing to sell.
The $R^{f}_{t}$ can come out of the expectation because there is no mystery about $p_{t+1}$ as there is \emph{no} $p_{t+1}$.
\end{frame}


\begin{frame}
\frametitle{Remember:}
\begin{figure}
\centering
\begin{minipage}{.5\textwidth}
  \centering
  \includegraphics[width=\textwidth]{fig_utility_vs_marginal_utility.eps}
\end{minipage}%
\begin{minipage}{.5\textwidth}
  \centering
  \includegraphics[width=\textwidth]{fig_utility_vs_marginal_utility_2.eps}
\end{minipage}
\end{figure}
\begin{itemize}[label={--}]
\item When consumption is high, marginal utility is low; and
\item when consumption is low, marginal utility is high
\end{itemize}
\end{frame}

\begin{frame}[label=sec-4-6]{Interpreting asset prices in an endowment economy\dots{} \\ \dots{}again}
\begin{align*}
\mathbb{E}_{t} \left[ R_{t+1} \right] = R^{f}_{t} \left( 1 - \mathbb{C} \text{ov} \left[ R_{t+1}, \frac{\beta u^{\prime}(c_{t+1})}{u^{\prime}(c_{t})} \right] \right)
\end{align*}

\begin{itemize}[label={--}]
\item When the covariance term is positive and positive, the asset pays off in high marginal-utility states, which is desirable
\item Drives up the price and down the return
\end{itemize}
\end{frame}



\subsection{Production}
\begin{frame}[label=sec-2-6]{Returns to scale}
A production function is homogeneous of degree $r$ if
\begin{align*}
f(aK,aN) = a^{r} f(K,N),
\end{align*}
for $a \in \mathbb{R}_{++}$, $K,N \in \mathbb{R}_{+}$.

A homogeneous production function exhibits
\begin{itemize}[label={--}]
\item increasing returns to scale if $r > $
\item decreasing returns to scale if $r < 1$
\item constant returns to scale if $r = 0$
\end{itemize}
\end{frame}

\begin{frame}[label=sec-2-6]{Returns to scale: Cobb--Douglas}
A Cobb--Douglas production function $F(K,N) = AK^{a}N^{b}$ is homoegenous of degree $a + b$:
\begin{align*}
F(\lambda K, \lambda N) &= A \left( \lambda K \right)^{a} \left( \lambda N \right)^{b} \\
&= \lambda^{a+b} \left\{ AK^{a}N^{b} \right\} \\
&= \lambda^{a+b} F(K,N)
\end{align*}
\end{frame}

\begin{frame}[label=sec-2-6]{Returns to scale: Additive}
Let $F(K,N) = K^{.5} + N^{.5}$. What are the returns to scale?

\begin{align*}
F(\lambda K, \lambda N) &= \left( \lambda K \right)^{.5} + \left( \lambda N \right)^{.5} \\
&= \lambda^{.5} \left[ K^{.5} + N^{.5} \right] \\
&= \lambda^{.5} F(K,N)
\end{align*}

$\implies$ doubling inputs \emph{does not} double production $\implies$ decreasing returns
\end{frame}

\begin{frame}[label=sec-2-6]{Returns to scale: Gin and tonic production}
Let $F(K,N) = \min \left\{ \lambda K, \lambda N \right\}$. I.e.,
\begin{itemize}[label={--}]
\item 8 parts gin
\item 1 part tonic
\end{itemize}
(for when I'm done with the review session)

\begin{align*}
F(\lambda K, \lambda N) &= \min \left\{ \lambda K, \lambda N \right\} \\
&= \lambda \min \left\{ K,N \right\}
\end{align*}

$\implies$ 16 parts gin + 2 parts water = 2 g\&t's $\implies$ constant returns
\end{frame}

\begin{frame}[label=sec-2-7]{Inputs are paid their marginal products}
Production side of an economy:
\begin{itemize}[label={--}]
\item Time is discrete and indexed by $t=0,1,\dots$
\item The final-good firm at any point in time
\begin{itemize}[label={--}]
\item Hires labor supplied by household members at rate $w_{t}$ per period
\item Rents capital at rate $R_{t}$ per period 
\end{itemize}
\end{itemize}

Firms solve
\begin{align*}
\max_{K,N} F(K,N) - R_{t}K - w_{t}N.
\end{align*}
\end{frame}

\begin{frame}[label=sec-2-7]{Inputs are paid their marginal products: \\ First-order conditions}
Firms' first-order conditions:
\begin{align*}
R_{t} &= F_{K}(K_{t},N_{t}) =: \text{MPK}_{t} \\
w_{t} &= F_{N}(K_{t},N_{t}) =: \text{MPN}_{t}
\end{align*}

\textcolor{RubineRed}{Fun fact}: If production is parameterized as Cobb--Douglas,
then these firms make \emph{no profits} in equilibrium.
\end{frame}

\subsection{User cost of capital}
\begin{frame}[label=sec-2-8]{Pricing a unit of capital: Capital-leasing firms}
Ponder: How much to pay for a unit of capital?
\begin{align*}
p_{t}^{K} &= \frac{R_{t+1}}{1+r} + \frac{(1-\delta)R_{t+2}}{(1+r)^{2}} +
\frac{(1-\delta)^{2}R_{t+3}}{(1+r)^{3}} \\
&= \frac{R_{t+1}}{1+r} + \frac{1-\delta}{1+r} \left[
\underbrace{\frac{R_{t+2}}{1+r} + \frac{(1-\delta)R_{t+3}}{(1+r)^{2}} + \cdots}_{p_{t+1}^{K}}
\right] \\
&= \frac{R_{t+1} + (1-\delta)p_{t+1}^{K}}{1+r}.
\end{align*}

$R_{t}$ is being paid by the firms renting capital, so
\begin{align*}
p_{t}^{K} = \frac{\text{MPK}_{t+1} + (1-\delta)p_{t+1}^{K}}{1+r}
\end{align*}
\end{frame}

\begin{frame}[label=sec-2-9]{Interpreting the pricing of a unit of capital}
\begin{align*}
p_{t}^{K} = \frac{\text{MPK}_{t+1} + (1-\delta)p_{t+1}^{K}}{1+r}
\end{align*}

Interpretation: You buy a unit of capital,
it comes online the following period (so discount future), and
you get to rent the unit of capital out---which then depreciates and then you get to sell the unit of capital.

\vspace{1em}

The derivation makes clear: Buying and holding the unit of capital provides no return over buying and selling.
\end{frame}

\begin{frame}[label=sec-2-9]{Interpreting the pricing of a unit of capital: \\ Story 2}
\begin{align*}
(1+r)p_{t}^{K} = \text{MPK}_{t+1} + (1-\delta)p_{t+1}^{K}
\end{align*}
Interpretation: Indifferent between the two strategies
\begin{itemize}[label={--}]
\item Invest $p_{t}^{K}$ and get $(1+r)p_{t}^{K}$ the following period
\item Buy a unit of capital and get
\begin{align*}
\text{MPK}_{t+1} + (1-\delta)p_{t+1}^{K}
\end{align*}
the following period
\end{itemize}
\end{frame}

\begin{frame}[label=sec-2-9]{No $t$'s and the user cost of capital}
\begin{align*}
(1+r)p^{K} &= \text{MPK} + (1-\delta)p^{K} \\
\therefore (r+\delta) p^{K} &= \text{MPK} = R
\end{align*}
and if $p^{K} = 1$ (like in the Solow model)
\begin{align*}
r = R - \delta
\end{align*}
i.e., $r$ is the marginal product less the depreciation
\end{frame}

\section{Two-period endowment economies}

\begin{frame}
\frametitle{Outline}
\tableofcontents[currentsection]
% currentsubsection,
% hideothersubsections,
% sectionstyle=show/hide,
% subsectionstyle=show]
\end{frame}

\begin{frame}
\frametitle{Economic environment}
\begin{itemize}[label={--}]
\item Time is discrete and indexed by $t=1,2$
\item The economy is populated by agents that make choices about consumption and saving and a government
\item Agents are endowed with labor income each period that is either spent on the consumption good or saved
\item The government faces an exogenous sequence of expenditures $\left\{ G_{1},G_{2} \right\}$
\item The expenditure levels can be motivated in a variety of ways
\item As \textcolor{Plum}{Pharaoh House} put it once: These are the pyramid-building requirements
\end{itemize}
\end{frame}

\begin{frame}
\frametitle{Material-balance condition}
\begin{itemize}[label={--}]
\item No production technology is available to the economy
\item $Y_{t} = C_{t} + I_{t} + G_{t} = C_{t} + G_{t} \text{ for } t=1,2$
\item Capital letters denote aggregate quantities
\end{itemize}
\end{frame}

\begin{frame}
\frametitle{The government}
\begin{itemize}[label={--}]
\item The government faces the exogenous pyramid-building requirements $\left\{ G_{1},G_{2} \right\}$
\item The government finances these expenditures by
\begin{itemize}[label={--}]
\item Levying lump-sum taxes $\left\{ T_{1},T_{2} \right\}$ and
\item Issuing debt $\left\{ B_{1},B_{2} \right\}$
\end{itemize}
\item I will use the convention that $B_{t} > 0$ is debt and $B_{t} < 0$ is government saving
\item The government faces borrowing cost $r_{1}$
\end{itemize}
\end{frame}

\begin{frame}
\frametitle{Government budget constraint}
The government's flow budget constraints are
\begin{align*}
G_{1} &\leq T_{1} + B_{1} \\
G_{2} &\leq T_{2} + B_{2} - (1-r_{1})B_{1}
\end{align*}
\begin{itemize}[label={--}]
\item The government is not going to waste money, so the equations hold with equality
\item The government cannot also hold debt at the end, so $B_{2} = 0$
\end{itemize}
\end{frame}

\begin{frame}
\frametitle{Unified government constraint}
\begin{align*}
G_{1} &= T_{1} + B_{1} \\
\therefore B_{1} &= G_{1} - T_{1} \\
\therefore G_{2} &= T_{2} - (1-r_{1}) \left( G_{1} - T_{1} \right),
\end{align*}
Thus
\begin{align*}
G_{1} + \frac{G_{2}}{1+r_{1}} = T_{1} + \frac{T_{2}}{1+r_{1}}
\end{align*}
\end{frame}

\begin{frame}
\frametitle{Agents' consumption and saving choices: Budget constraint}
\begin{itemize}[label={--}]
\item Agents are endowed each period with labor income $\left\{ y_{1},y_{2} \right\}$
\item Face the sequence of flow budget constraints
\begin{align*}
c_{1} + s_{1} &= y_{1} - T_{1} \\
c_{2} + s_{2} &= y_{2} - T_{2} + (1+r_{1})s_{1}
\end{align*}
\item In period $1$, after-tax income is divided between consumption and saving
\item In period $2$, after-tax income plus any saving is divided between consumption and saving
\item $s_{2} = 0$
\end{itemize}
\end{frame}

\begin{frame}
\frametitle{Agents' consumption and saving choices: Budget constraint}
Unified budget constraint:
\begin{align*}
s_{1} &= y_{1} - T_{1} - c_{1} \\
\therefore c_{2} + s_{2} &= y_{2} - T_{2} + (1+r_{1})s_{1} \\
&= y_{2} - T_{2} + (1+r_{1}) \left( y_{1} - T_{1} - c_{1} \right) \\
\therefore \frac{c_{2}}{1+r_{1}} &= \frac{y_{2} - T_{2}}{1+r_{1}} + y_{1} - T_{1} - c_{1},
\end{align*}
Thus
\begin{align*}
c_{1} + \frac{c_{2}}{1+r_{1}} = y_{1} + \frac{y_{2}}{1+r_{1}} - \left( T_{1} + \frac{T_{2}}{1+r_{1}} \right)
\end{align*}
\begin{itemize}[label={--}]
\item Fast and loose with demographics
\item $T_{1}$ is aggregate transfers
\item $c_{1}$ is individual-level consumption
\item Set $N = 1$
\end{itemize}
\end{frame}

\begin{frame}
\frametitle{Plotting budget sets}
Re-arrange:
\begin{align*}
C_{2} &= (1+r_{1})\left[ Y_{1} + \frac{Y_{2}}{1+r_{1}} - \left( T_{1} + \frac{T_{2}}{1+r_{1}} \right) \right] - (1+r_{1})C_{1} \\
&= (1+r_{1})\left[ Y_{1} + \frac{Y_{2}}{1+r_{1}} - \left( G_{1} + \frac{G_{2}}{1+r_{1}} \right) \right] - (1+r_{1})C_{1}
\end{align*}
The government is just taking the endowment\dots
\end{frame}

\begin{frame}
\frametitle{Plotting budget sets}
\centerline{\includegraphics[width=0.85\textwidth]{figs/fig_econ402_2_period_gov_C_choice.eps}}
\end{frame}

\begin{frame}
\frametitle{Consumption and saving choices}
The representative agent has standard preferences over streams of consumption:
\begin{align*}
\mathcal{U}(C_{1},C_{2}) = \ln C_{1} + \beta \ln C_{2}.
\end{align*}
Agents maximize $\mathcal{U}$ subject the the intertemporal budget constraint 

\vspace{1em}

The consumption Euler equation is
\begin{align*}
\frac{1}{c_{1}} &= \beta (1+r_{1}) \frac{1}{c_{2}} \\
\therefore \frac{c_{2}}{1+r_{1}} &= \beta c_{1}.
\end{align*}
\end{frame}

\begin{frame}
\frametitle{Consumption function}
Euler equation combined with the intertemporal budget constraint yields
\begin{align*}
C_{1}(r;Y,G) = \frac{1}{1+\beta} \left[ Y_{1} + \frac{Y_{2}}{1+r_{1}} - \left( G_{1} + \frac{G_{2}}{1+r_{1}} \right) \right]
\end{align*}
\end{frame}

\begin{frame}
\frametitle{Balancing material}
\begin{itemize}[label={--}]
\item The bond-market-clearing condition requires $B_{t} = S_{t}$
\item Government-issued debt equals saving
\item In a closed economy, aggregate saving is $0$
\item So
\begin{align*}
0 &= \underbrace{S_{t}}_{\text{private saving}} \underbrace{- B_{t}}_{\text{public saving}} \\
\therefore S_{t} &= B_{t}.
\end{align*}
\end{itemize}
\end{frame}

\begin{frame}
\frametitle{When $B_{1} = 0$}
\begin{itemize}[label={--}]
\item When $B_{1} = 0$ the government finances consumption through lump-sum transfers
\item Then $G_{1} = T_{1}$ and $G_{2} = T_{2}$
\item Recall there is a representative agent
\begin{align*}
S_{1} &= Y_{1} - T_{1} - C_{1} \\
&= Y_{1} - G_{1} - C_{1}
\end{align*}
\item And using $C_{1}$
\begin{align*}
S_{1} &= Y_{1} - G_{1} - \frac{1}{1+\beta} \left[ Y_{1} + \frac{Y_{2}}{1+r_{1}} - \left( G_{1} + \frac{G_{2}}{1+r_{1}} \right) \right] \\
&= \frac{\beta}{1+\beta}Y_{1} - \frac{\beta}{1+\beta} G_{1} - \frac{Y_{2}}{(1+\beta)(1+r_{1})} + \frac{G_{2}}{(1+\beta)(1+r_{1})}
\end{align*}
\end{itemize}
\end{frame}

\begin{frame}
\frametitle{Solving for $r_1$}
Re-arranging:
\begin{align*}
1+r_{1} = \frac{Y_{2}-G_{2}}{\beta \left( Y_{1}-G_{1} \right) - (1+\beta)S_{1}}
\end{align*}
Plot the relationship\dots
\end{frame}

\begin{frame}
\frametitle{Determining $r_{1}$ in pictures}
\centerline{\includegraphics[width=\textwidth]{figs/fig_econ402_2_period_gov_interest_rate.eps}}
\end{frame}

\begin{frame}
\frametitle{Story: $G_{1}$ increase}
What happens when $G_{1}$ increases?
\begin{itemize}[label={--}]
\item This is an endowment economy, so $Y_{1}$ is fixed
\item And $C_{1} = Y_{1} - G_{1}$ goes down---consumption is ``crowded out'' (one-for-one in the case where $B_{1} = 0$)
\item The agent is poorer
\item They want to smooth out consumption and pull some forward from period $2$
\item In order to maintain aggregate saving equal to $0$, the interest rate must rise to discourage too much smoothing
\end{itemize}
\end{frame}

\begin{frame}
\frametitle{Story: $G_{2}$ increase}
What happens when $G_{2}$ increases?
\begin{itemize}[label={--}]
\item The agent knows they will be poorer in the future, which entices them to save
\item For any level of the interest rate, the agent saves more---this is depicted in figure
\item In order to maintain zero aggregate saving, the interest rate must rise to discourage this effect
\end{itemize}
\end{frame}

\label{sec-3}
\section{Solow}

\begin{frame}
\frametitle{Outline}
\tableofcontents[currentsection]
% currentsubsection,
% hideothersubsections,
% sectionstyle=show/hide,
% subsectionstyle=show]
\end{frame}

\begin{frame}
\frametitle{There are two types of questions that can be asked}

\begin{enumerate}[label=(\roman*)]
\item\label{item:3} Solve for $K^{\star}$
\begin{itemize}[label={--}]
\item You did this on Midterm Exam II
\end{itemize}
\item Growth rate of $K$, $Y$, $C$, $I$ along the balanced growth path
\begin{itemize}[label={--}]
\item Balanced growth path: $\frac{K_{t+1}}{K_{t}} = \frac{K_{t+2}}{K_{t+1}}$
\item You did this on EC3 (it can't get harder than that)
\end{itemize}
\end{enumerate}

% \begin{enumerate}
% \item\label{item:1} 
% \item\label{item:2} 

% \end{enumerate}
\end{frame}

\section{IS--LM}

\begin{frame}
\frametitle{Outline}
\tableofcontents[currentsection]
% currentsubsection,
% hideothersubsections,
% sectionstyle=show/hide,
% subsectionstyle=show]
\end{frame}

\subsection{IS--LM in normal times}
\label{sec-5}
\begin{frame}[label=sec-5-1]{Increase $M$: Effects of a monetary expansion}

\end{frame}
\begin{frame}[label=sec-5-2]{Increase $G$}
\end{frame}
\begin{frame}[label=sec-5-3]{ZLB}
\end{frame}

\subsection{IS--LM and a liquidity trap}

\subsection{Ben Bernanke: Open-mouth operations}


\begin{frame}
\frametitle{Ben Bernanke and the Sirens}
\centerline{\includegraphics[width=\textwidth]{figs/Seductive-Sirens-of-Greek-Mythology.jpg}}
\end{frame}

\begin{frame}
\frametitle{Bernanke's open-mouth operations}
\begin{itemize}[label={--}]
\item Ben Bernanke: ``No matter what, keep interest rates low''
\item $\implies \pi^{e}$ increase
\item No matter what: war,
discover oil (or gold for that matter),
Patriots win yet another Super Bowl (House goes nuts and buys Tom Brady jerseys to wear every day)
\item Promise to keep interest rates at 0
\item (Imagine you buy a bond expecting low interest rates and then Bernanke reneges, raising interest rates)
\end{itemize}
\end{frame}

\begin{frame}
\frametitle{Open-mouth operations}
\begin{align*}
Y &= C(Y,r) + I(r) + G \\
&= C(Y, i - \pi^{e}) + I(i-\pi^{e}) + G
\end{align*}
\begin{itemize}[label={--}]
\item $\pi^{e} \uparrow$ like decreasing $r$
\item $I_{r} < 0$ and $C_{r} < 0$
\item $I$ and $C$ inrease when $\pi^{e} \uparrow$
\item Shifts out IS curve
\end{itemize}
\end{frame}


\section{Mundell--Fleming}
\label{sec-6}

\begin{frame}
\frametitle{Outline}
\tableofcontents[currentsection]
% currentsubsection,
% hideothersubsections,
% sectionstyle=show/hide,
% subsectionstyle=show]
\end{frame}

\begin{frame}[label=sec-6-1]{Opening trade with a country with low $K$}
\begin{itemize}[label={--}]
\item When $p^{k}_{t} = 1$, $r_{t} = r$: user-cost of capital requires
\begin{align*}
r = R - \delta = \text{MPK} - \delta
\end{align*}
\item Low $K \implies$ high $\text{MPK}$ $\implies$ high $r$
\end{itemize}
\end{frame}

\begin{frame}[label=sec-6-1]{Opening trade with a country with low $K$}
Question: \textcolor{RubineRed}{Starting from blanced trade ($NX = 0$),
what if open trade with a very large country with low level of capital?}
\begin{itemize}[label={--}]
\item What happens?
\item Earn higher rate of return abroad
\item (This is how Professor House could ask the question---combines Parts I and III)
\end{itemize}
\end{frame}

\begin{frame}
\frametitle{Opening trade with a country with high $r^{\star}$}
\centerline{\includegraphics[width=\textwidth]{rwr_lecture_notes/econ402_w2018_lecture10_opening_trade_low_K.png}}
\end{frame}

\begin{frame}
\frametitle{Opening trade with a country with high $r^{\star}$: Story}
\begin{itemize}[label={--}]
\item In the figure: $S > I$
\item In a closed economy: $S = I$
\item How can you save more than you invest? Why isn't your saving contributing to the domestic capital stock?
\item $NX = S - I > 0$
\item Don't do domestic projects; send money abroad to earn high rate of return
\item Trade surplus = net lenders in world (two-country, in this example) financial markets
\item \textcolor{RubineRed}{Take advantage of high rate of return and lend abroad}
\end{itemize}
\end{frame}

\begin{frame}
\frametitle{Mundell--Fleming}
\begin{itemize}[label={--}]
\item Small, open economy?
\item Small = decisions have negligible effects on prices
\item Open = Borrow and lend as much as desired in world financial markets
\item $r = r^{\star}$
\end{itemize}
\end{frame}

\begin{frame}
\frametitle{Mundell--Fleming: IS$^{\star}$}
Market for goods and services:
\begin{align*}
Y = C(Y-T, r^{\star})  + I(r^{\star}) + G + \underbrace{NX(e)}_{-} ,
\end{align*}
where $e$ is the nominal exchange rate.

\vspace{1em}

IS$^{\star}$ is downward sloping:
\begin{align*}
\frac{de}{dY} = \frac{1-\text{MPC}}{NX_{e}} < 0 \text{ as } 0 < \text{MPC} < 1
\end{align*}
\end{frame}

% \begin{frame}
% \frametitle{Mundell--Fleming: LM$^{\star}$}
% Supply, $M/P$, equals demand, $L(r,Y)$, in the money market requires
% \begin{align*}
% M/P = L(r^{\star},Y)
% \end{align*}
% \end{frame}

\begin{frame}
\frametitle{Mundell--Fleming: LM$^{\star}$}
% \begin{centering}
\centering
\includegraphics[height=0.85\textheight]{rwr_lecture_notes/econ402_w2018_lecture11_mundell_fleming_LM_star.png}
% \end{centering}
\end{frame}

\begin{frame}
\frametitle{Mundell--Fleming: IS$^{\star}$--LM$^{\star}$}
\centering
\includegraphics[width=\textwidth]{rwr_lecture_notes/econ402_w2018_lecture11_mundell_fleming_IS_LM_star.png}
\end{frame}



% \begin{frame}
% \frametitle{Mundell--Fleming: $e$}
% \begin{itemize}[label={--}]
% \item $e$ is the nominal exchange rate
% \item $\varepsilon$ is the real exchange rate
% \item $\varepsilon = \frac{e P}{P^{\star}}$
% \item $P$ is the domestic price
% \item $P^{\star}$ is the foreign price
% \item Prices fixed in IS--LM framework
% \end{itemize}
% \end{frame}

\begin{frame}
\frametitle{Mundell--Fleming: $NX(e)$ depends negatively on $e$}
Market for goods and services:
\begin{align*}
Y = C(Y-T, r^{\star})  + I(r^{\star}) + G + \underbrace{NX(e)}_{-}.
\end{align*}
\begin{itemize}[label={--}]
\item $e \uparrow$ means more \pounds per \$
\item Domestic goods are more expensive as $e \uparrow$
\item Exports go down
\item Imports go up
\item $NX(e) \downarrow$
\end{itemize}
\end{frame}

\begin{frame}[label=sec-6-3]{Mundell--Fleming: Fixed vs floating exchange rates}
\begin{itemize}[label={--}]
\item Floating exchange rates: The exchange rate is determined by market forces and responds to economic conditions
\item Fixed exchange rates: The central bank announces a value for the exchange rate and stands ready to buy and sell their domestic currency to keep the exchanges rate at the announced level
\end{itemize}
\end{frame}

\begin{frame}
\frametitle{Fixed exchange rate: Fiscal policy}
\begin{itemize}[label={--}]
\item Small, open economy
\item Interest rate = $r^{\star}$
\item Fixed exchange rates---central bank willing to buy and sell currency to keep $e = e^{\star}$
\item Fiscal expansion: $G \uparrow$
\end{itemize}
\end{frame}

\begin{frame}[label=sec-6-2]{Fixed exchange rate: Monetary expansion}
\begin{itemize}[label={--}]
\item Small, open economy
\item Interest rate = $r^{\star}$
\item Fixed exchange rates---central bank willing to buy and sell currency to keep $e = e^{\star}$
\item Monetary expansion, $M \uparrow$
\end{itemize}
\end{frame}
\begin{frame}[label=sec-6-2]{Fixed exchange rate: Monetary expansion}
\includegraphics[width=\textwidth]{rwr_lecture_notes/econ402_w2018_lecture11_M_fixed_exchange_LM_curve.png}

$\implies$ LM$^{\star}$ curve shifts to the right

\end{frame}
\begin{frame}[label=sec-6-2]{Fixed exchange rate: Monetary expansion}
\includegraphics[]{rwr_lecture_notes/econ402_w2018_lecture11_M_fixed_exchange.png}
\end{frame}
\begin{frame}
\frametitle{Fixed exchange rate: Monetary expansion}
\begin{itemize}[label={--}]
\item At $A$, $e = \$ / \pounds$ is too low
\item Too many dollars; ie, dollars are cheap
\item The central bank exchanges \pounds for \$, bidding up \pounds
\item (The central bank runs down it supply of foreign reserves)
\item Exchanging \pounds for \$ pushes down $M$
\item The LM curve shifts back to LM$_{1}$ from LM$_{2}$
\item \textcolor{RubineRed}{There is no effect on $Y$}
\end{itemize}
\end{frame}

% \section{Facts}
% \label{sec-7}
% \begin{frame}[label=sec-7-1]{Current US Budget Issues}
% \end{frame}
% \begin{frame}[label=sec-7-2]{Economic Growth}
% \end{frame}
% \begin{frame}[label=sec-7-3]{Business Cycle Analysis}
% \end{frame}
% \begin{frame}[label=sec-7-4]{Mundell--Fleming}
% \begin{block}{The nominal exchange rate is in units of yen per dollar}
% \end{block}
% \begin{block}{T/F: The top 0.1\% hold roughly 0.1\% of wealth (F)}
% \end{block}
% \begin{block}{T/F: The uncovered interest rate parity $\Delta e_{t+1}$}
% \end{block}
% \begin{block}{T/F: The Taylor rule specifies that the Fed responds more than one-for-one to the interest rate}
% \end{block}
% \begin{block}{T/F: The Federal Reserve sets a target or a range for the FFR}
% \end{block}
% \begin{block}{T/F: How long is the average expansion?}
% \end{block}
% \begin{block}{T/F: Maiden Lane got around sec 14 of the Federal Reserve Act, setting up an LLC}
% \end{block}
% \begin{block}{T/F: Maiden Lane is the street is the Fed Bank of New York is on?}
% \end{block}
% \begin{block}{T/F: The Federal Reserve purchased mortgage backed securities as part of the Mortgage Backed Securities (MBS) Purchase Program}
% \end{block}
% \begin{block}{T/F: The largest part of unconventional monetary policy was MBS Purchase Program}
% \end{block}
% \begin{block}{T/F: Uncovered carry trade}
% \end{block}
% \begin{block}{T/F: How do you measure $\alpha$ in the data? Labor share.}
% \end{block}
% \begin{block}{T/F:}
% \end{block}
% \end{frame}


% Emacs 24.5.1 (Org mode 8.2.10)
\end{document}