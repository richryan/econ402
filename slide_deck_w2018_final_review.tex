% Created 2018-04-17 Tue 14:04
\documentclass[presentation,dvipsnames]{beamer}
% \usepackage[dvipsnames]{xcolor}
% \usepackage{fixltx2e}
% \usepackage{graphicx}
% \usepackage{longtable}
% \usepackage{float}
% \usepackage{wrapfig}
% \usepackage{rotating}
% \usepackage[normalem]{ulem}
\usepackage{amsmath,amssymb}
% \usepackage{marvosym}
% \usepackage{wasysym}
% \usepackage{amssymb}
\usepackage{enumitem}
\usepackage[T1]{fontenc}
\usepackage{textcomp}           % to get \pounds to work in math mode

\renewcommand{\pounds}{\text{\textsterling}{ }}

% \usepackage[math]{iwona}
% \usepackage{listings}
\usetheme{default}
\author{}
\date{April 19 \& 20, 2018}
\title{W2018 ECON 402 Final Review}
\hypersetup{
  pdfkeywords={},
  pdfsubject={},
  pdfcreator={Emacs 24.5.1 (Org mode 8.2.10)}}
\begin{document}

\maketitle
\begin{frame}{Outline}
\setcounter{tocdepth}{1}
\tableofcontents
\end{frame}

\section{Outline}
\label{sec-1}
\section{Important concepts}
\label{sec-2}
\begin{frame}[label=sec-2-1]{Intertemporal choice}
\end{frame}
\begin{frame}[label=sec-2-2]{Consumption Euler equation}
\end{frame}
\begin{frame}[label=sec-2-3]{Sums and partial sums}
\end{frame}
\begin{frame}[label=sec-2-4]{PIH}
\end{frame}
\begin{frame}[label=sec-2-5]{Fundamentals of asset pricing}
\end{frame}
\begin{frame}[label=sec-2-6]{Returns to scale}
\end{frame}
\begin{frame}[label=sec-2-7]{Inputs are paid their marginal products}
\end{frame}
\begin{frame}[label=sec-2-8]{Pricing a unit of capital}
\end{frame}
\begin{frame}[label=sec-2-9]{Interpreting the pricing of a unit of capital}
\begin{block}{Remark 4.6}
\end{block}
\begin{block}{Remark 4.7}
\end{block}
\end{frame}
\section{Two-period endowment economies}
\label{sec-3}
\section{Solow}
\label{sec-4}
\begin{frame}[label=sec-4-1]{$K_{\star}$}
\end{frame}
\begin{frame}[label=sec-4-2]{Balanced growth}
\end{frame}
\section{IS--LM}
\label{sec-5}
\begin{frame}[label=sec-5-1]{Increase $M$: Effects of a monetary expansion}

\end{frame}
\begin{frame}[label=sec-5-2]{Increase $G$}
\end{frame}
\begin{frame}[label=sec-5-3]{ZLB}
\end{frame}
\section{Mundell--Fleming}
\label{sec-6}

\begin{frame}[label=sec-6-1]{Opening trade with a country with low $K$}
\end{frame}

\begin{frame}
\frametitle{Mundell--Fleming}
\begin{itemize}[label={--}]
\item Small, open economy?
\item Borrow and lend as much as desired in world financial markets
\item Small = decisions have negligible effects on prices ($r = r^{\star}$)
\end{itemize}
\end{frame}

\begin{frame}
\frametitle{Mundell--Fleming: IS$^{\star}$}
Market for goods and services:
\begin{align*}
Y = C(Y-T, r^{\star})  + I(r^{\star}) + G + \underbrace{NX(e)}_{-} ,
\end{align*}
where $e$ is the nominal exchange rate.

\vspace{1em}

IS$^{\star}$ is downward sloping:
\begin{align*}
\frac{de}{dY} = \frac{1-\text{MPC}}{NX_{e}} < 0 \text{ as } 0 < \text{MPC} < 1
\end{align*}
\end{frame}

% \begin{frame}
% \frametitle{Mundell--Fleming: LM$^{\star}$}
% Supply, $M/P$, equals demand, $L(r,Y)$, in the money market requires
% \begin{align*}
% M/P = L(r^{\star},Y)
% \end{align*}
% \end{frame}

\begin{frame}
\frametitle{Mundell--Fleming: LM$^{\star}$}
% \begin{centering}
\centering
\includegraphics[height=0.85\textheight]{rwr_lecture_notes/econ402_w2018_lecture11_mundell_fleming_LM_star.png}
% \end{centering}
\end{frame}

\begin{frame}
\frametitle{Mundell--Fleming: IS$^{\star}$--LM$^{\star}$}
\centering
\includegraphics[width=\textwidth]{rwr_lecture_notes/econ402_w2018_lecture11_mundell_fleming_IS_LM_star.png}
\end{frame}



% \begin{frame}
% \frametitle{Mundell--Fleming: $e$}
% \begin{itemize}[label={--}]
% \item $e$ is the nominal exchange rate
% \item $\varepsilon$ is the real exchange rate
% \item $\varepsilon = \frac{e P}{P^{\star}}$
% \item $P$ is the domestic price
% \item $P^{\star}$ is the foreign price
% \item Prices fixed in IS--LM framework
% \end{itemize}
% \end{frame}

\begin{frame}
\frametitle{Mundell--Fleming: $NX(e)$ depends negatively on $e$}
\begin{itemize}[label={--}]
\item $e \uparrow$ means more \pounds per \$
\item Domestic goods are more expensive as $e \uparrow$
\item Exports go down
\item Imports go up
\item $NX(e) \downarrow$
\end{itemize}
\end{frame}

\begin{frame}[label=sec-6-3]{Mundell--Fleming: Fixed vs floating exchange rates}
\begin{itemize}[label={--}]
\item Floating exchange rates: The exchange rate is determined by market forces and responds to economic conditions
\item Fixed exchange rates: The central bank announces a value for the exchange rate and stands ready to buy and sell their domestic currency to keep the exchanges rate at the announced level
\end{itemize}
\end{frame}

\begin{frame}
\frametitle{Fixed exchange rate: Fiscal policy}
\begin{itemize}[label={--}]
\item Small, open economy
\item Interest rate = $r^{\star}$
\item Fixed exchange rates---central bank willing to buy and sell currency to keep $e = e^{\star}$
\item Fiscal expansion: $G \uparrow$
\end{itemize}
\end{frame}

\begin{frame}[label=sec-6-2]{Fixed exchange rate: Monetary expansion}
\begin{itemize}[label={--}]
\item Small, open economy
\item Interest rate = $r^{\star}$
\item Fixed exchange rates
\item Monetary expansion
\end{itemize}
\end{frame}
\begin{frame}[label=sec-6-2]{Fixed exchange rate: Monetary expansion}
\includegraphics[width=\textwidth]{rwr_lecture_notes/econ402_w2018_lecture11_M_fixed_exchange_LM_curve.png}

$\implies$ LM$^{\star}$ curve shifts to the right

\end{frame}
\begin{frame}[label=sec-6-2]{Fixed exchange rate: Monetary expansion}
\includegraphics[]{rwr_lecture_notes/econ402_w2018_lecture11_M_fixed_exchange.png}
\end{frame}
\begin{frame}
\frametitle{Fixed exchange rate: Monetary expansion}
\begin{itemize}[label={--}]
\item At $A$, $e = \$ / \pounds$ is too low
\item Too many dollars; ie, dollars are cheap
\item The central bank exchanges \pounds for \$, bidding up \pounds
\item (The central bank runs down it supply of foreign reserves)
\item Exchanging \pounds for \$ pushes down $M$
\item The LM curve shifts back to LM$_{1}$ from LM$_{2}$
\item \textcolor{RubineRed}{There is no effect on $Y$}
\end{itemize}
\end{frame}

\section{Facts}
\label{sec-7}
\begin{frame}[label=sec-7-1]{Current US Budget Issues}
\end{frame}
\begin{frame}[label=sec-7-2]{Economic Growth}
\end{frame}
\begin{frame}[label=sec-7-3]{Business Cycle Analysis}
\end{frame}
\begin{frame}[label=sec-7-4]{Mundell--Fleming}
\begin{block}{The nominal exchange rate is in units of yen per dollar}
\end{block}
\begin{block}{T/F: The top 0.1\% hold roughly 0.1\% of wealth (F)}
\end{block}
\begin{block}{T/F: The uncovered interest rate parity $\Delta e_{t+1}$}
\end{block}
\begin{block}{T/F: The Taylor rule specifies that the Fed responds more than one-for-one to the interest rate}
\end{block}
\begin{block}{T/F: The Federal Reserve sets a target or a range for the FFR}
\end{block}
\begin{block}{T/F: How long is the average expansion?}
\end{block}
\begin{block}{T/F: Maiden Lane got around sec 14 of the Federal Reserve Act, setting up an LLC}
\end{block}
\begin{block}{T/F: Maiden Lane is the street is the Fed Bank of New York is on?}
\end{block}
\begin{block}{T/F: The Federal Reserve purchased mortgage backed securities as part of the Mortgage Backed Securities (MBS) Purchase Program}
\end{block}
\begin{block}{T/F: The largest part of unconventional monetary policy was MBS Purchase Program}
\end{block}
\begin{block}{T/F: Uncovered carry trade}
\end{block}
\begin{block}{T/F: How do you measure $\alpha$ in the data? Labor share.}
\end{block}
\begin{block}{T/F:}
\end{block}
\end{frame}
% Emacs 24.5.1 (Org mode 8.2.10)
\end{document}