% Created 2018-04-17 Tue 14:04
\documentclass[presentation,dvipsnames]{beamer}
% \usepackage[dvipsnames]{xcolor}
% \usepackage{fixltx2e}
% \usepackage{graphicx}
% \usepackage{longtable}
% \usepackage{float}
% \usepackage{wrapfig}
% \usepackage{rotating}
% \usepackage[normalem]{ulem}
\usepackage{amsmath,amssymb,amsthm}
\newtheorem{prop}{Proposition}
% \usepackage{marvosym}
% \usepackage{wasysym}
% \usepackage{amssymb}
\usepackage{enumitem}
\usepackage[T1]{fontenc}
\usepackage{textcomp}           % to get \pounds to work in math mode

\renewcommand{\pounds}{\text{\textsterling}{ }}

% \usepackage[math]{iwona}
% \usepackage{listings}
\usetheme{default}
\author{}
\date{April 19 \& 20, 2018}
\title{W2018 ECON 402 Final Review}
\hypersetup{
  pdfkeywords={},
  pdfsubject={},
  pdfcreator={Emacs 24.5.1 (Org mode 8.2.10)}}
\begin{document}

\maketitle
\begin{frame}{Outline}
\setcounter{tocdepth}{1}
\tableofcontents
\end{frame}

\section{Outline}
\label{sec-1}
\section{Important concepts}

\subsection{Intertemporal choice}
\label{sec-2}

\begin{frame}[label=sec-2-1]{Intertemporal choice: Environment}
The agent makes decisions about consumption and saving to maximize utility
\begin{align*}
\mathcal{U}(c_{1},c_{2}) = \ln c_{1} + \beta \ln c_{2},
\end{align*}
where $\beta$ is the subjective discount factor and choices are made subject to
\begin{align*}
c_{1} + s_{1} &= y_{1} \\
c_{2} + s_{2} &= y_{2} + (1+r_{1})s_{1}.
\end{align*}
\begin{itemize}[label={--}]
\item $s_{2} = 0$ because there is no period 3
\item Borrow and lend freely ($s_{1}$ can be $<0$, $=0$, or $>0$):
\begin{align*}
s_{1} &= \frac{c_{2}-y_{2}}{1+r_{1}} \\
\therefore c_{1} + \frac{c_{2}}{1+r_{1}} &= y_{1} + \frac{y_{2}}{1+r_{1}}
\end{align*}
\item Intertemporal budget constraint or ``unified'' budget constraint
\end{itemize}
\end{frame}

\begin{frame}[label=sec-2-1]{Intertemporal choice: The choice part}
\centerline{\includegraphics[width=\textwidth]{figs/econ402_figure_2_2.PNG}}
\end{frame}

\begin{frame}[label=sec-2-1]{Intertemporal choice: Math behind the choice part}
\begin{align*}
\mathcal{U} = \ln \left( y_{1} + \frac{y_{2}}{1+r_{1}} - \frac{c_{2}}{1+r_{1}} \right) +
\beta \ln c_{2}
\end{align*}
First-order condition for $c_{2}$:
\begin{align*}
0 &= \frac{1}{c_{1}} \frac{-1}{1+r_{1}} + \beta \frac{1}{c_{2}} \\
\therefore \frac{1}{c_{1}} &= \beta (1+r) \frac{1}{c_{2}}
\end{align*}
In the figure:
\begin{align*}
- \frac{1 / c_{1}}{\beta / c_{2}} &= - (1+r_{1}) \\
\text{MRS} &= \text{price ratio}
\end{align*}
\end{frame}

\begin{frame}
\frametitle{Intertemporal choice: Regions of borrowing and lending}
\centerline{\includegraphics[width=\textwidth]{figs/econ402_figure_2_3.PNG}}
\end{frame}

\begin{frame}
\frametitle{Lenders and the interest rate rises}
\begin{prop}
\label{prop:1}
If an agent is a lender and the interest rate rises,
then they will remain a lender.
\end{prop}
You can always consumer your endowment:
\centerline{\includegraphics[width=\textwidth]{figs/econ402_figure_2_4.PNG}}
\end{frame}

\begin{frame}
\frametitle{Lenders and the interest rate rises}
\begin{prop}
\label{prop:1}
If an agent is a lender and the interest rate rises,
then they will remain a lender.
\end{prop}

\begin{proof}
\begin{itemize}[label={--}]
\item You can always consumer your endowment (gives rotation)
\item Is it possible that after the increase in the interest rate the agent chooses a point to the right of the endowment?
\item \emph{Absolutely not}. Revealed preference
\item They could have afforded all these points initially
\item Must be to the left of $y_{1} \implies$ remain lender
\end{itemize}
\end{proof}
\end{frame}

\subsection{Consumption Euler equation, generally}
\begin{frame}[label=sec-2-2]{Consumption Euler equation, generally}
Infinitely-lived agent chooses streams of consumption to maximize
\begin{align*}
\mathcal{U}(c_{0},c_{1},\dots) = \sum\limits_{t=0}^{\infty} \beta^{t} u(c_{t})
\end{align*}
subject to the Tom Brady budget constraint:
\begin{align*}
\sum\limits_{t=0}^{\infty} \frac{c_{t}}{(1+r)^{t}} = \sum\limits_{t=0}^{\infty} \frac{y_{t}}{(1+r)^{t}} +
\underbrace{A_{0}}_{\text{D.~Trump Jr.~initial assets}}
\end{align*}
\end{frame}

\subsection{Consumption Euler equation}
\begin{frame}[label=sec-2-2]{Consumption Euler equation, generally}
\begin{align*}
\text{MU}(c_{t}) := \frac{\partial \mathcal{U}}{\partial c_{t}} = \beta^{t} u^{\prime}(c_{t})
\end{align*}
Tangency condition (think of the two-period case):
\begin{align*}
\text{MU}(c_{t}) &= (1+r) \text{MU}(c_{t+1}) \\
\therefore \beta^{t} u^{\prime}(c_{t}) &= (1+r) \beta^{t+1}u^{\prime}(c_{t+1}) \\
\therefore u^{\prime}(c_{t}) &= (1+r) \beta u^{\prime}(c_{t+1})
\end{align*}
\begin{itemize}[label={--}]
\item Could be $r_{t}$ but have to deal carefully with the Tom Brady budget constraint
\item If log preferences:
\begin{align*}
\frac{1}{c_{t}} = (1+r)\beta \frac{1}{c_{t+1}}
\end{align*}
\end{itemize}
\end{frame}

\begin{frame}[label=sec-2-2]{Consumption Euler equation, generally}
\framesubtitle{\dots or more generally, if Professor House wants to be funny}
\begin{align*}
u^{\prime}(c_{t}) = \beta (1+r_{1}) \mathbb{E}_{t} \left[ u^{\prime}(c_{t+1}) \right],
\end{align*}
where $\mathbb{E}_{t}$ is the expectations sign,
which takes into account of all information available at time $t$
(hence the $t$ subscript, if that wasn't abundantly clear)
\end{frame}

\subsection{Partial sums}
\begin{frame}[label=sec-2-3]{Sums and partial sums (before we can do PIH)}
\begin{prop}
\label{prop:2}
For $|\alpha| < 1$,
\begin{align*}
1 + \alpha + \alpha^{2} + \alpha^{3} + \cdots = \sum\limits_{t=0}^{\infty} \alpha^{t} = \frac{1}{1-\alpha}.
\end{align*}
\end{prop}
\begin{proof}
We are interested in the sum $S$:
\begin{align*}
S &= 1 + \alpha + \alpha^{2} + \alpha^{3} + \cdots \\
\implies \alpha S &= \alpha + \alpha^{2} + \alpha^{3} + \cdots \\
\implies S - \alpha S &= (1-\alpha)S = 1.
\end{align*}
Thus
\begin{align*}
S = \frac{1}{1-\alpha}.
\end{align*}
\end{proof}
\end{frame}

\begin{frame}[label=sec-2-3]{Sums and partial sums (before we can do PIH)}
\begin{prop}
\label{prop:2}
For $|\alpha| < 1$ and $n \in \mathbb{N}$
\begin{align*}
\sum\limits_{j=0}^{n} \alpha^{j} = 1+\alpha + \alpha^{2} + \cdots + \alpha^{n} = \frac{1-\alpha^{n+1}}{1-\alpha}.
\end{align*}
\end{prop}
\begin{proof}
Use the same steps as above.
\end{proof}
\end{frame}

\subsection{PIH}
\begin{frame}[label=sec-2-4]{Perfect consumption smoothing}
\framesubtitle{With $\beta (1+r) = 1$}
From
\begin{align*}
\frac{1}{c_{t}} &= \beta (1+r) \frac{1}{c_{t+1}} \\
\implies \frac{c_{t+1}}{c_{t}} &= \beta (1+r) \\
&= 1.
\end{align*}
Therefore
\begin{align*}
c_{t+1} = c_{t} = c.
\end{align*}
\end{frame}

\begin{frame}
\frametitle{Use $c_{t+1} = c_{t} = c$ in the lifetime budget constraint}
Use our geometric-series result:
\begin{align*}
\sum\limits_{t=0}^{\infty} \frac{c_{t}}{(1+r)^{t}} &=
\sum\limits_{t=0}^{\infty} \frac{y_{t}}{(1+r)^{t}} + A_{0} \\
\therefore \sum\limits_{t=0}^{\infty} \frac{c}{(1+r)^{t}} &=
\sum\limits_{t=0}^{\infty} \frac{y_{t}}{(1+r)^{t}} + A_{0} \\
\therefore \frac{c}{1- \frac{1}{1+r}} &=
\sum\limits_{t=0}^{\infty} \frac{y_{t}}{(1+r)^{t}} + A_{0} \\
\therefore \frac{c}{\frac{r}{1+r}} &=
\sum\limits_{t=0}^{\infty} \frac{y_{t}}{(1+r)^{t}} + A_{0}.
\end{align*}
Thus
\begin{align*}
c = \frac{r}{1+r} \left[ \sum\limits_{t=0}^{\infty} \frac{y_{t}}{(1+r)^{t}} + A_{0} \right].
\end{align*}
\end{frame}

\begin{frame}
\frametitle{Perfect consumption smoothing}
\begin{align*}
c = \frac{r}{1+r} \left[ \sum\limits_{t=0}^{\infty} \frac{y_{t}}{(1+r)^{t}} + A_{0} \right]
\end{align*}
If $y_{t} = y$, then
\begin{align*}
c &= \frac{r}{1+r} \left[ \sum\limits_{t=0}^{\infty} \frac{y}{(1+r)^{t}} + A_{0} \right] \\
&= \frac{r}{1+r} \left[ \frac{y}{\frac{r}{1+r}} + A_{0} \right] \\
&= y + \frac{r}{1+r} A_{0}.
\end{align*}

Strong motive to smooth consumption.
\end{frame}

\begin{frame}
\frametitle{Strong motive to smooth consumption}
Ben Bernankie: That can't be good:
\centerline{\includegraphics[width=\textwidth]{figs/econ402_FRED_real_personal_consumption_expenditures.png}}
\end{frame}

\subsection{Fundamentals of asset pricing}
\begin{frame}[label=sec-2-5]{Fundamentals of asset pricing}
\end{frame}

\subsection{Returns to scale}
\begin{frame}[label=sec-2-6]{Returns to scale}
\end{frame}

\subsection{Inputs are paid their marginal products}
\begin{frame}[label=sec-2-7]{Inputs are paid their marginal products}
\end{frame}

\subsection{Pricing a unit of capital}
\begin{frame}[label=sec-2-8]{Pricing a unit of capital}
\end{frame}

\subsection{Interpreting the pricing of a unit of capital}
\begin{frame}[label=sec-2-9]{Interpreting the pricing of a unit of capital}
\begin{itemize}
\item You puchase a unit of capital
\item You rent out a 
\end{itemize}
\end{frame}

\begin{block}{Remark 4.6}
\end{block}
\begin{block}{Remark 4.7}
\end{block}
\section{Two-period endowment economies}
\label{sec-3}
\section{Solow}
\label{sec-4}
\begin{frame}[label=sec-4-1]{$K_{\star}$}
\end{frame}
\begin{frame}[label=sec-4-2]{Balanced growth}
\end{frame}
\section{IS--LM}
\label{sec-5}
\begin{frame}[label=sec-5-1]{Increase $M$: Effects of a monetary expansion}

\end{frame}
\begin{frame}[label=sec-5-2]{Increase $G$}
\end{frame}
\begin{frame}[label=sec-5-3]{ZLB}
\end{frame}
\section{Mundell--Fleming}
\label{sec-6}

\begin{frame}[label=sec-6-1]{Opening trade with a country with low $K$}
\begin{itemize}[label={--}]
\item When $p^{k}_{t} = 1$, $r_{t} = r$: user-cost of capital requires
\begin{align*}
r = R - \delta = \text{MPK} - \delta
\end{align*}
\item Low $K \implies$ high $\text{MPK}$ $\implies$ high $r$
\end{itemize}
\end{frame}

\begin{frame}[label=sec-6-1]{Opening trade with a country with low $K$}
Question: \textcolor{RubineRed}{Starting from blanced trade ($NX = 0$),
what if open trade with a very large country with low level of capital?}
\begin{itemize}[label={--}]
\item What happens?
\item Earn higher rate of return abroad
\item (This is how Professor House could ask the question---combines Parts I and III)
\end{itemize}
\end{frame}

\begin{frame}
\frametitle{Opening trade with a country with high $r^{\star}$}
\centerline{\includegraphics[width=\textwidth]{rwr_lecture_notes/econ402_w2018_lecture10_opening_trade_low_K.png}}
\end{frame}

\begin{frame}
\frametitle{Opening trade with a country with high $r^{\star}$: Story}
\begin{itemize}[label={--}]
\item In the figure: $S > I$
\item In a closed economy: $S = I$
\item How can you save more than you invest? Why isn't your saving contributing to the domestic capital stock?
\item $NX = S - I > 0$
\item Don't do domestic projects; send money abroad to earn high rate of return
\item Trade surplus = net lenders in world (two-country, in this example) financial markets
\item \textcolor{RubineRed}{Take advantage of high rate of return and lend abroad}
\end{itemize}
\end{frame}

\begin{frame}
\frametitle{Mundell--Fleming}
\begin{itemize}[label={--}]
\item Small, open economy?
\item Small = decisions have negligible effects on prices
\item Open = Borrow and lend as much as desired in world financial markets
\item $r = r^{\star}$
\end{itemize}
\end{frame}

\begin{frame}
\frametitle{Mundell--Fleming: IS$^{\star}$}
Market for goods and services:
\begin{align*}
Y = C(Y-T, r^{\star})  + I(r^{\star}) + G + \underbrace{NX(e)}_{-} ,
\end{align*}
where $e$ is the nominal exchange rate.

\vspace{1em}

IS$^{\star}$ is downward sloping:
\begin{align*}
\frac{de}{dY} = \frac{1-\text{MPC}}{NX_{e}} < 0 \text{ as } 0 < \text{MPC} < 1
\end{align*}
\end{frame}

% \begin{frame}
% \frametitle{Mundell--Fleming: LM$^{\star}$}
% Supply, $M/P$, equals demand, $L(r,Y)$, in the money market requires
% \begin{align*}
% M/P = L(r^{\star},Y)
% \end{align*}
% \end{frame}

\begin{frame}
\frametitle{Mundell--Fleming: LM$^{\star}$}
% \begin{centering}
\centering
\includegraphics[height=0.85\textheight]{rwr_lecture_notes/econ402_w2018_lecture11_mundell_fleming_LM_star.png}
% \end{centering}
\end{frame}

\begin{frame}
\frametitle{Mundell--Fleming: IS$^{\star}$--LM$^{\star}$}
\centering
\includegraphics[width=\textwidth]{rwr_lecture_notes/econ402_w2018_lecture11_mundell_fleming_IS_LM_star.png}
\end{frame}



% \begin{frame}
% \frametitle{Mundell--Fleming: $e$}
% \begin{itemize}[label={--}]
% \item $e$ is the nominal exchange rate
% \item $\varepsilon$ is the real exchange rate
% \item $\varepsilon = \frac{e P}{P^{\star}}$
% \item $P$ is the domestic price
% \item $P^{\star}$ is the foreign price
% \item Prices fixed in IS--LM framework
% \end{itemize}
% \end{frame}

\begin{frame}
\frametitle{Mundell--Fleming: $NX(e)$ depends negatively on $e$}
Market for goods and services:
\begin{align*}
Y = C(Y-T, r^{\star})  + I(r^{\star}) + G + \underbrace{NX(e)}_{-}.
\end{align*}
\begin{itemize}[label={--}]
\item $e \uparrow$ means more \pounds per \$
\item Domestic goods are more expensive as $e \uparrow$
\item Exports go down
\item Imports go up
\item $NX(e) \downarrow$
\end{itemize}
\end{frame}

\begin{frame}[label=sec-6-3]{Mundell--Fleming: Fixed vs floating exchange rates}
\begin{itemize}[label={--}]
\item Floating exchange rates: The exchange rate is determined by market forces and responds to economic conditions
\item Fixed exchange rates: The central bank announces a value for the exchange rate and stands ready to buy and sell their domestic currency to keep the exchanges rate at the announced level
\end{itemize}
\end{frame}

\begin{frame}
\frametitle{Fixed exchange rate: Fiscal policy}
\begin{itemize}[label={--}]
\item Small, open economy
\item Interest rate = $r^{\star}$
\item Fixed exchange rates---central bank willing to buy and sell currency to keep $e = e^{\star}$
\item Fiscal expansion: $G \uparrow$
\end{itemize}
\end{frame}

\begin{frame}[label=sec-6-2]{Fixed exchange rate: Monetary expansion}
\begin{itemize}[label={--}]
\item Small, open economy
\item Interest rate = $r^{\star}$
\item Fixed exchange rates---central bank willing to buy and sell currency to keep $e = e^{\star}$
\item Monetary expansion, $M \uparrow$
\end{itemize}
\end{frame}
\begin{frame}[label=sec-6-2]{Fixed exchange rate: Monetary expansion}
\includegraphics[width=\textwidth]{rwr_lecture_notes/econ402_w2018_lecture11_M_fixed_exchange_LM_curve.png}

$\implies$ LM$^{\star}$ curve shifts to the right

\end{frame}
\begin{frame}[label=sec-6-2]{Fixed exchange rate: Monetary expansion}
\includegraphics[]{rwr_lecture_notes/econ402_w2018_lecture11_M_fixed_exchange.png}
\end{frame}
\begin{frame}
\frametitle{Fixed exchange rate: Monetary expansion}
\begin{itemize}[label={--}]
\item At $A$, $e = \$ / \pounds$ is too low
\item Too many dollars; ie, dollars are cheap
\item The central bank exchanges \pounds for \$, bidding up \pounds
\item (The central bank runs down it supply of foreign reserves)
\item Exchanging \pounds for \$ pushes down $M$
\item The LM curve shifts back to LM$_{1}$ from LM$_{2}$
\item \textcolor{RubineRed}{There is no effect on $Y$}
\end{itemize}
\end{frame}

\section{Facts}
\label{sec-7}
\begin{frame}[label=sec-7-1]{Current US Budget Issues}
\end{frame}
\begin{frame}[label=sec-7-2]{Economic Growth}
\end{frame}
\begin{frame}[label=sec-7-3]{Business Cycle Analysis}
\end{frame}
\begin{frame}[label=sec-7-4]{Mundell--Fleming}
\begin{block}{The nominal exchange rate is in units of yen per dollar}
\end{block}
\begin{block}{T/F: The top 0.1\% hold roughly 0.1\% of wealth (F)}
\end{block}
\begin{block}{T/F: The uncovered interest rate parity $\Delta e_{t+1}$}
\end{block}
\begin{block}{T/F: The Taylor rule specifies that the Fed responds more than one-for-one to the interest rate}
\end{block}
\begin{block}{T/F: The Federal Reserve sets a target or a range for the FFR}
\end{block}
\begin{block}{T/F: How long is the average expansion?}
\end{block}
\begin{block}{T/F: Maiden Lane got around sec 14 of the Federal Reserve Act, setting up an LLC}
\end{block}
\begin{block}{T/F: Maiden Lane is the street is the Fed Bank of New York is on?}
\end{block}
\begin{block}{T/F: The Federal Reserve purchased mortgage backed securities as part of the Mortgage Backed Securities (MBS) Purchase Program}
\end{block}
\begin{block}{T/F: The largest part of unconventional monetary policy was MBS Purchase Program}
\end{block}
\begin{block}{T/F: Uncovered carry trade}
\end{block}
\begin{block}{T/F: How do you measure $\alpha$ in the data? Labor share.}
\end{block}
\begin{block}{T/F:}
\end{block}
\end{frame}
% Emacs 24.5.1 (Org mode 8.2.10)
\end{document}