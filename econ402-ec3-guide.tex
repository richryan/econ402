% =====================================================================
% Program: econ-402-ec3.tex
% Purpose: Suggested solutions to the second extra credit.
%
% E-mail: RichRyan@umich.edu
% =====================================================================
\documentclass[12pt]{pracjourn_rwr}
% \documentclass[12pt]{article}

\usepackage[dvipsnames]{xcolor}
\usepackage{url}
\urlstyle{same}
\usepackage{booktabs}
\usepackage{graphicx}
\usepackage{amsmath,amsthm,amssymb}
\usepackage{amsxtra}
\usepackage{natbib}
\usepackage[left=1in,right=1in,top=1in,bottom=1in]{geometry}
\usepackage{alltt}
\input{ee.sty}
\usepackage{listings}
% \usepackage[inline]{showlabels}
\usepackage{paralist} % inline lists (enumeration)
\usepackage{mathabx} % widebar

% Set new counter to start everything off with a "2" for section 2
\newcounter{rich}
\setcounter{rich}{3}
\newcounter{ryan}
\setcounter{ryan}{5}
\renewcommand{\theryan}{\Alph{ryan}\arabic{rich}}

% Change figure numbering
\usepackage{chngcntr}
\counterwithin{figure}{ryan}

% \newtheorem{theorem}{Theorem}[chapter]
% \newtheorem{lemma}{Lemma}[chapter]
\newtheorem{theorem}{Theorem}[rich]
\newtheorem{lemma}{Lemma}[rich]

% \newtheoremstyle{definition}% Name
% {3pt}% Space above
% {3pt}% Space below
% {}% Body font
% {}% Indent amount
% {\bfseries}% Theorem head font
% {:}% Punctuation after theorem head
% {.5em}% Space after theorem head
% {}% Theorem head spec (can be left empty, meaning "normal")

% \setcounter{chapter}{2}
\theoremstyle{definition}
\newtheorem{definition}{Definition}[rich]
\newtheorem{example}{Example}[rich]
\newtheorem{exercise}{Exercise}[rich]
\newtheorem{prop}{Proposition}[rich]
\newtheorem{assumption}{Assumption}[rich]
\newtheorem{fact}{Fact}[rich]

\theoremstyle{remark}
\newtheorem{remark}{Remark}[ryan]
% \newtheorem{remark}{Remark}[chapter]
% \newtheorem{fact}{Fact}[chapter]

% === Change to homework numbering ===
\renewcommand{\thesection}{E\arabic{rich}.\arabic{section}}

\title{Grading Guide for Extra Credit \#3}
% \author{Rich Ryan}
% \email{RichRyan@umich.edu}
%% \TPJrevision{2013}{08}{03} % The editors ask you to update this for each %% revision submitted
% \TPrevision{3}{1}{2015}
% \license{\texttt{econ-402-ec2.tex}} % May be used if you want a copyright notice at the bottom right of the first page
% \abstract{Time-varying prices of capital.}
% \newinfo\datestart{Date Started}
% \newinfo\daterevised{Date Revised} % See the pracjourn class for how to do this
% \datestart{March 7, 2018}
% \daterevised{March 13, 2018}


% === Spacing in the table of contents
\usepackage{tocloft}
\renewcommand{\cftsecnumwidth}{3em}
\renewcommand{\cftsubsecnumwidth}{5em}
\renewcommand{\cftsecleader}{\cftdotfill{\cftdotsep}}
% Set TOC depth
\setcounter{tocdepth}{2}


\begin{document}
\maketitle


There are 30 points to be awarded for EC \#3.

\section{R\&D in the Solow growth model: 15 points}
\begin{itemize}
% === PART A ===
\item Part a), 3 points total. Attempt = 2 points. Award 1 additional point if state that all variables grow at a constant rate.
\item Part b), 6 points total. Attempt = 3 points. Award 3 additional points for
\begin{align*}
\frac{K_{t+1}}{K_{t}} = 1 + \phi z.
\end{align*}
\item Part c), 3 points total. Attempt = 2 points. Award 1 additional point if say that it is positively related.
\item Part d), 3 points total. Attempt = 3 points.
\end{itemize}

\section{A linear IS/LM model: 15 points}

\begin{itemize}
\item Part a), 5 points total. Attempt = 3 points. Award 2 additional points for
\begin{align*}
\frac{dY}{dG} = \frac{1}{1-\text{MPC}}.
\end{align*}
\item Part b), 5 points total. Attempt = 3 points. Award 2 additional points for
\begin{align*}
\frac{dY}{dG} = \frac{1}{1-\text{MPC} + \frac{\varepsilon}{\phi}}.
\end{align*}
\item Part c), 5 points total. Attempt = 3 points.  Award 2 additional points for
\begin{align*}
\frac{1}{1-\text{MPC} + \frac{\varepsilon}{\phi}} < \frac{1}{1-\text{MPC}}.
\end{align*}
\end{itemize}

% \bibliography{c:/Users/rwr/bibliography/bibliography}
% \bibliographystyle{c:/Users/rwr/bibliography/bostonfed}
\end{document}

%%% Local Variables: 
%%% mode: latex
%%% TeX-engine: xetex
%%% End: