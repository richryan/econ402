% =====================================================================
% Program: econ402-2-period-gov.tex
% Purpose: Two-period endowment economy with government.
%
% E-mail: RichRyan@umich.edu
% =====================================================================
\documentclass[12pt]{pracjourn_rwr}
% \documentclass[12pt]{article}

\usepackage[dvipsnames]{xcolor}
\usepackage{url}
\urlstyle{same}
\usepackage{booktabs}
\usepackage{subfig}
\usepackage{graphicx}
\usepackage{amsmath,amsthm,amssymb}
\usepackage{amsxtra}
\usepackage{natbib}
\usepackage[left=1in,right=1in,top=1in,bottom=1in]{geometry}
\usepackage{alltt}
\input{ee.sty}
\usepackage{listings}
% \usepackage[inline]{showlabels}
\usepackage{paralist} % inline lists (enumeration)
\usepackage{mathabx} % widebar

% Change "References" to "Reference"
\renewcommand\refname{Reference}

\theoremstyle{remark}
\newtheorem{remark}{Remark}

\title{Two-period endowment economies with government}
% \author{Rich Ryan}
\email{RichRyan@umich.edu}
%% \TPJrevision{2013}{08}{03} % The editors ask you to update this for each %% revision submitted
% \TPrevision{3}{1}{2015}
\license{\texttt{econ-402-2-period-gov.tex}} % May be used if you want a copyright notice at the bottom right of the first page
\abstract{Government expenditure is added to a two-period endowment economy.
Examples of crowindg in and out are given.}
\newinfo\datestart{Date Started}
\newinfo\daterevised{Date Revised} % See the pracjourn class for how to do this
\datestart{March 17, 2018}
\daterevised{March 19, 2018}

\begin{document}
\maketitle

\tableofcontents

\section{Economic environment and the material-balance condition}

Time is discrete and indexed by $t=1,2$.
The economy is populated by agents that make choices about consumption and saving and a government.
Agents are endowed with labor income each period that is either spent on the consumption good or saved.
The government faces an exogenous sequence of expenditures $\left\{ G_{1},G_{2} \right\}$.
The expenditure levels can be motivated in a variety of ways.
As Pharaoh House put it once: These are the pyramid-building requirements.

No production technology is available to the economy.
The material-balance condition therefore requires
\begin{equation}
\label{eq:1}
Y_{t} = C_{t} + I_{t} + G_{t} = C_{t} + G_{t} \text{ for } t=1,2;
\end{equation}
where capital letters denote aggregate quantities.
% Because this is a closed economy proposition 2.4 requires that $S_{t} 0$ for $t=1,2$.

The economic environment is an extension of the economic environment considered in H2.2,
available \url{https://richryan.github.io/econ402/notes.html}.

\section{The government}

The government faces the exogenous pyramid-building requirements $\left\{ G_{1},G_{2} \right\}$.
The government finances these expenditures by levying lump-sum taxes $\left\{ T_{1},T_{2} \right\}$ and
issuing debt $\left\{ B_{1},B_{2} \right\}$.
I will use the convention that $B_{t} > 0$ is debt and $B_{t} < 0$ is government saving.
The government faces borrowing cost $r_{1}$.

The government's flow budget constraints are
\begin{align}
\label{eq:2}
G_{1} &\leq T_{1} + B_{1} \\
\label{eq:3}
G_{2} &\leq T_{2} + B_{2} - (1-r_{1})B_{1}.
\end{align}
Equation \eqref{eq:2} says that government consumption in period $1$ is less than or equal to revenue raised through lump-sum taxes plus any debt issued.
Equation \eqref{eq:3} says that government consumption in period $2$ is less than or equal to lump-sum taxes raised in the second period less any debt payments the government is required to make.

\begin{remark}
\label{remark:1}
The government is not going to waste money, so equations \eqref{eq:2} and \eqref{eq:3} will hold with equality.
The government cannot also hold debt at the end, so $B_{2} = 0$.
\end{remark}

Using remark \ref{remark:1}, then,
eliminating $B_{1}$ from equations \eqref{eq:2} and \eqref{eq:3} leads to the government's intertemporal budget constraint
\begin{align*}
G_{1} &= T_{1} + B_{1} \\
\therefore B_{1} &= G_{1} - T_{1} \\
\therefore G_{2} &= T_{2} - (1-r_{1}) \left( G_{1} - T_{1} \right),
\end{align*}
where the last line uses \eqref{eq:3}.
Thus
\begin{equation}
\label{eq:4}
G_{1} + \frac{G_{2}}{1+r_{1}} = T_{1} + \frac{T_{2}}{1+r_{1}}.
\end{equation}
Euqation \eqref{eq:4} requires that the present discounted value of government consumption equal
the present discounted value revenue raised through lump-sum transers.

\section{Agents' consumption and saving choices}

\subsection{Budget constraint}
Agents are endowed each period with labor income $\left\{ y_{1},y_{2} \right\}$.
They divide this income between consumption and saving.
They face the sequence of flow budget constraints
\begin{align}
\label{eq:5}
c_{1} + s_{1} &= y_{1} - T_{1} \\
\label{eq:6}
c_{2} + s_{2} &= y_{2} - T_{2} + (1+r_{1})s_{1}.
\end{align}
Equation \eqref{eq:5} says that in period $1$ after-tax income is divided between consumption and saving.
Equation \eqref{eq:6} says that in period $2$ after-tax income plus any saving is divided between consumption and saving.

Remark \ref{remark:1} applies for the household as well.
There is no sense in saving in the second period, so $s_{2} = 0$.
Combining equations \eqref{eq:5} and \eqref{eq:6} leads to the lifetime budget constraint.
Starting with equation \eqref{eq:5}
\begin{align*}
s_{1} &= y_{1} - T_{1} - c_{1} \\
\therefore c_{2} + s_{2} &= y_{2} - T_{2} + (1+r_{1})s_{1} \\
&= y_{2} - T_{2} + (1+r_{1}) \left( y_{1} - T_{1} - c_{1} \right) \\
\therefore \frac{c_{2}}{1+r_{1}} &= \frac{y_{2} - T_{2}}{1+r_{1}} + y_{1} - T_{1} - c_{1},
\end{align*}
where the last line uses $s_{2} = 0$.
This equation can be re-arranged to read
\begin{equation}
\label{eq:7}
c_{1} + \frac{c_{2}}{1+r_{1}} = y_{1} + \frac{y_{2}}{1+r_{1}} - \left( T_{1} + \frac{T_{2}}{1+r_{1}} \right).
\end{equation}
Equation \eqref{eq:7} is the unified budget constraint.

\begin{remark}
\label{remark:2}
I've already been a little fast and loose with the demographics.
$T_{1}$ is aggregate transfers,
so I've implicitly been assuming there is a single representative consumer.
It would be a straightforward extension to have $N$ consumers of different types like we did in H2.2.
But going forward, I'll make the representative-consumer framework explicit by using capital letters,
which denote aggregate quantities.

By the way,
$Y_{t}$ is simply the period-$t$ endowment---just like the economies we studied in the notes in sections 1, 2, and 3 and in
H2.2.
\end{remark}

Equation \eqref{eq:7} can be re-arranged to plot:
\begin{align*}
C_{2} &= (1+r_{1})\left[ Y_{1} + \frac{Y_{2}}{1+r_{1}} - \left( T_{1} + \frac{T_{2}}{1+r_{1}} \right) \right] - (1+r_{1})C_{1} \\
&= (1+r_{1})\left[ Y_{1} + \frac{Y_{2}}{1+r_{1}} - \left( G_{1} + \frac{G_{2}}{1+r_{1}} \right) \right] - (1+r_{1})C_{1},
\end{align*}
where the second line uses \eqref{eq:4}.
The agent's intertemporal budget constraint is depicted in figure \ref{fig:C-choice}.

\begin{figure}[htbp]
% \centerline{\includegraphics[0.85\textwidth]{figs/fig_econ402_2_period_gov_C_choice.eps}}
\centerline{\includegraphics[width=0.85\textwidth]{figs/fig_econ402_2_period_gov_C_choice.eps}}
\caption[]{\label{fig:C-choice} Intertemporal budget constraint faced by the representative agent.}
\end{figure}

\subsection{Consumption and saving choices}

The representative agent has standard preferences over streams of consumption:
\begin{align*}
\mathcal{U}(C_{1},C_{2}) = \ln C_{1} + \beta \ln C_{2}.
\end{align*}
Agents maximize $\mathcal{U}$ subject the the intertemporal budget constraint \eqref{eq:7}.

The consumption Euler equation is
\begin{align*}
\frac{1}{c_{1}} &= \beta (1+r_{1}) \frac{1}{c_{2}} \\
\therefore \frac{c_{2}}{1+r_{1}} &= \beta c_{1}.
\end{align*}
The latter combined with the intertemporal budget constraint in \eqref{eq:7} yields
\begin{align*}
C_{1} + \frac{C_{2}}{1+r_{1}} &= Y_{1} + \frac{Y_{2}}{1+r_{1}} - \left( G_{1} + \frac{G_{2}}{1+r_{1}} \right) \\
\therefore (1+\beta)C_{1} &= Y_{1} + \frac{Y_{2}}{1+r_{1}} - \left( G_{1} + \frac{G_{2}}{1+r_{1}} \right),
\end{align*}
or
\begin{equation}
\label{eq:8}
C_{1}(r;Y,G) = \frac{1}{1+\beta} \left[ Y_{1} + \frac{Y_{2}}{1+r_{1}} - \left( G_{1} + \frac{G_{2}}{1+r_{1}} \right) \right].
\end{equation}


\section{Determining the interest rate through national accounting}

\subsection{Balancing material}
The material-balance condition requires
\begin{align*}
Y_{t} = C_{t} + G_{t}.
\end{align*}
There is, however, a saving technology available to agents.
The bond-market-clearing condition requires
\begin{equation}
\label{eq:9}
B_{t} = S_{t};
\end{equation}
that is, government-issued debt must equal saving.
The debt is bought by the household.

We know aggregate saving is zero (from proposition 2.4).
There are now two components:
\begin{align*}
0 &= \underbrace{S_{t}}_{\text{private saving}} \underbrace{- B_{t}}_{\text{public saving}} \\
\therefore S_{t} &= B_{t}.
\end{align*}
(Recall that negative debt is government saving.)

\subsection{When $B_{1} =0$}

When $B_{1} = 0$ the government finances consumption through lump-sum transfers.
Looking at \eqref{eq:2} and \eqref{eq:3}, $G_{1} = T_{1}$ and $G_{2} = T_{2}$.

When this is the case, we're back to case where $S_{1} = 0$,
which pins down the interest rate.

Using equation \eqref{eq:8} (recall that there is a representative agent),
\begin{align*}
S_{1} &= Y_{1} - T_{1} - C_{1} \\
&= Y_{1} - G_{1} - C_{1},
\end{align*}
where the last line uses $B_{1} = 0$ and thus
\begin{align*}
S_{1} &= Y_{1} - G_{1} - \frac{1}{1+\beta} \left[ Y_{1} + \frac{Y_{2}}{1+r_{1}} - \left( G_{1} + \frac{G_{2}}{1+r_{1}} \right) \right] \\
&= \frac{\beta}{1+\beta}Y_{1} - \frac{\beta}{1+\beta} G_{1} - \frac{Y_{2}}{(1+\beta)(1+r_{1})} + \frac{G_{2}}{(1+\beta)(1+r_{1})}.
\end{align*}
Re-arraning yields
\begin{align*}
\frac{Y_{2}-G_{2}}{(1+\beta)(1+r_{1})} &= \frac{\beta}{1+\beta} \left( Y_{1}-G_{1} \right) - S_{1} \\
\therefore Y_{2}-G_{2} &= (1+r_{1}) (1+\beta) \left( \frac{\beta}{1+\beta} \left( Y_{1}-G_{1} \right) - S_{1} \right) \\
&= (1+r_{1}) \left[ \beta \left( Y_{1}-G_{1} \right) - (1+\beta)S_{1} \right]
\end{align*}
Thus
\begin{equation}
\label{eq:10}
1+r_{1} = \frac{Y_{2}-G_{2}}{\beta \left( Y_{1}-G_{1} \right) - (1+\beta)S_{1}}
\end{equation}
Figure \ref{fig:interest-rate} depicts the relationship in equation \eqref{eq:10}.

What happens when $G_{1}$ increases?
Mechanically, the effect can be read off of equation \eqref{eq:10}:
equilibrium $r_{1}$ increases.
Why?
This is an endowment economy, so $Y_{1}$ is fixed.
And $C_{1} = Y_{1} - G_{1}$ goes down---consumption is ``crowded out'' (one-for-one in the case where $B_{1} = 0$).
The agent is poorer.
They want to smooth out consumption and pull some forward from period $2$.
In order to maintain aggregate saving equal to $0$, the interest rate must rise to discourage too much smoothing.

What happens when $G_{2}$ increases?
The agent knows they will be poorer in the future,
which entices them to save.
For any level of the interest rate, the agent saves more---this is depicted in figure \ref{fig:interest-rate}.
In order to maintain zero aggregate saving, the interest rate must fall to discourage this effect.
(Because here $C_{1} = Y_{1} - G_{1}$, consumption isn't ``crowded in'' exactly;
but if $B_{1} > 0$, then you could see how consumption would be ``crowded in.''
This explains some of the red bullet points in \texttt{Lecture 10\_handouts}.)

\begin{remark}
\label{remark:3}
This isn't really Ricardian equivalence---it's mechanical.
\end{remark}


\begin{figure}[htbp]
\centerline{\includegraphics[]{figs/fig_econ402_2_period_gov_interest_rate.eps}}
\caption[]{\label{fig:interest-rate} Determination of the interest rate.}
\end{figure}

\bibliography{c:/Users/rwr/bibliography/bibliography}
\bibliographystyle{c:/Users/rwr/bibliography/bostonfed}
\end{document}

%%% Local Variables: 
%%% mode: latex
%%% TeX-engine: xetex
%%% End: